% !TEX program = xelatex
% !Mode:: "TeX:UTF-8"
\documentclass[color=purple,openany]{textbook-cn}
\makeatletter
%%设计封面%%
\newcommand{\ifetdef}{
\maxsizebox{\paperwidth}{!}{\ifdefvoid{\@englishtitle}{}{\@englishtitle}}
}
\newcommand{\makecover}{
\tikzset{coverup node/.style={text=ColorA,anchor=north west,inner sep=0mm},
coverdown node/.style={text=white,anchor=west,inner sep=0mm},
opacity node/.style={anchor=north west,font=\FontSize{100pt}\bfseries\sffamily,text=ColorB!25}
}
\begin{titlepage}
\begin{tikzpicture}[remember picture,overlay]
\coordinate(A)at(current page.north west);
\coordinate(B)at(current page.north east);
\coordinate(C)at(current page.south east);
\coordinate(D)at(current page.south west);
\coordinate(E)at([yshift=5.25cm]current page.west);
\coordinate(F)at([yshift=5.25cm]current page.east);
\coordinate(G)at($0.5*(E)+0.5*(C)$);
%%背景1%%
\draw[ColorB!5,fill=ColorB!5,sharp corners](D)rectangle(B);
%%半透明文字%%
\node[opacity node,text opacity=1.0](O_1)at(A){\ifetdef};
\foreach \x[evaluate=\x as \botpoint using int(\x-1),
evaluate=\x as \opa using 1.2-0.2*\x] in {2,...,10}{
\node[opacity node,text opacity=\opa](O_\x)at(O_\botpoint.south west){\ifetdef};}
%%背景2%%
\draw[ColorA!80,fill=ColorA!80](E)rectangle(C);
%%封面图%%
\node[inner sep=0mm,anchor=center](CI)at(G){\ifdefvoid{\@coverimage}{}{\includegraphics[width=\paperwidth,height=\paperheight/2+5.25cm]{\@coverimage}}};
%%标题%%
\node[font=\FontSize{55pt}\bfseries\sffamily,coverup node](T)at(current page text area.north west){\@title};
%%英文标题%%
\node[font=\FontSize{20pt}\bfseries\sffamily,coverup node](ET)at([yshift=-0.5cm]T.south west){\ifdefvoid{\@englishtitle}{}{\@englishtitle}};
%%副标题%%
\node[font=\FontSize{30pt}\bfseries\sffamily,coverup node](ST)at([yshift=-0.5cm]ET.south west){\ifdefvoid{\@subtitle}{}{\@subtitle}};
%%英文副标题%%
\node[font=\FontSize{15pt}\bfseries\sffamily,coverup node](EST)at([yshift=-0.5cm]ST.south west){\ifdefvoid{\@englishsubtitle}{}{\@englishsubtitle}};
%%作者%%
\node[font=\FontSize{15pt}\bfseries\sffamily,coverdown node](A)at([yshift=4.0cm]current page text area.west){主编\quad\@author};
%%出版社%%
\node[coverdown node](PL)at(current page text area.south west){\ifdefvoid{\@presslogo}{}{\includegraphics[width=0.75cm]{\@presslogo}}};
\node[font=\FontSize{15pt}\bfseries\sffamily,coverdown node](PN)at([xshift=0.25cm]PL.east){\ifdefvoid{\@pressname}{}{\@pressname}};
\end{tikzpicture}
\end{titlepage}}

%%设计封底%%
\newcommand{\ifestdef}{
\maxsizebox{\paperwidth}{!}{\ifdefvoid{\@englishsubtitle}{}{\@englishsubtitle}}
}
\NewDocumentCommand{\makeback}{oo}{
\tikzset{backup node/.style={text=ColorA,anchor=north east,inner sep=0mm,text width=\textwidth},
backdown node/.style={text=white,anchor=west,inner sep=0mm},
opacity node/.style={anchor=north east,font=\FontSize{100pt}\bfseries,text=ColorB!25}}
\begin{titlepage}
\begin{tikzpicture}[remember picture,overlay]
\coordinate(A)at(current page.north west);
\coordinate(B)at(current page.north east);
\coordinate(C)at(current page.south east);
\coordinate(D)at(current page.south west);
\coordinate(E)at([yshift=5.25cm]current page.west);
\coordinate(F)at([yshift=5.25cm]current page.east);
\coordinate(G)at($0.5*(E)+0.5*(C)$);
%%背景1%%
\draw[ColorB!5,fill=ColorB!5,sharp corners](D)rectangle(B);
%%半透明文字%%
\node[opacity node,text opacity=1.0](O_1)at(B){\ifestdef};
\foreach \x[evaluate=\x as \botpoint using int(\x-1),
evaluate=\x as \opa using 1.2-0.2*\x] in {2,...,10}{
\node[opacity node,text opacity=\opa](O_\x)at(O_\botpoint.south east){\ifestdef};}
%%背景2%%
\draw[ColorA!80,fill=ColorA!80](E)rectangle(C);
%%封底图%%
\node[inner sep=0mm,anchor=center](BI)at(G){\ifdefvoid{\@backimage}{}{\scalebox{1}[1]{\includegraphics[width=\paperwidth,height=\paperheight/2+5.25cm]{\@backimage}}}};
%%条形码%%
\IfValueTF{#2}{
\node[draw=white,fill=white,inner xsep=0mm,anchor=south east](BC)at([yshift=-0.5cm]current page text area.south east){\includegraphics[width=5.0cm]{#2}};}{\relax}
%%说明文字%%
\node[font=\FontSize{12pt}\bfseries\sffamily,backup node](BT)at(current page text area.north east){\hspace*{2.0em}#1};
%%书籍系列%%
\node[font=\FontSize{15pt}\bfseries\sffamily,backdown node](S)at([yshift=4.0cm]current page text area.west){\ifdefvoid{\@series}{}{\@series}};
%%书籍系列%%
\node[font=\FontSize{15pt}\bfseries\sffamily,backdown node](V)at([yshift=-0.5cm]S.south west){\ifdefvoid{\@version}{}{第{\@version}版}};
\end{tikzpicture}
\end{titlepage}}

%%设计书脊%%
\NewDocumentCommand{\makespine}{O{1.0cm}}{
\newlength{\SpineWidth}
\setlength{\SpineWidth}{#1}
\tikzset{spine title/.style={font=\FontSize{20pt}\bfseries\sffamily,text=white,text width=1.0cm,align=center},
spine subtitle/.style={font=\FontSize{15pt}\bfseries\sffamily,text=white,text width=1.0cm,align=center},
spine englishtitle/.style={font=\FontSize{20pt}\bfseries\sffamily,text=ColorA,inner sep=0mm}}
\begin{titlepage}
\begin{tikzpicture}[remember picture,overlay]
\coordinate(A)at(current page.north);
\coordinate(B)at(current page.south);
\coordinate(C)at([yshift=5.25cm]current page.center);
%%背景%%
\draw[ColorB!5,fill=ColorB!5]
([xshift=-\SpineWidth/2]A)rectangle([xshift=\SpineWidth/2]B);
\draw[ColorA!80,fill=ColorA!80]
([xshift=-\SpineWidth/2]C)rectangle([xshift=\SpineWidth/2]B);
%%书脊英文标题%%
\node[spine englishtitle,anchor=north](ET)at([yshift=-0.5cm]A){\ifdefvoid{\@englishtitle}{}{\rotatebox[origin=c]{-90}{\maxsizebox{\paperheight/2-5.25cm}{\SpineWidth}{\@englishtitle}}}};
%%书脊标题%%
\node[spine title,anchor=north](T)at([yshift=-0.5cm]C){\@title};
%%书脊副标题%%
\node[spine subtitle,anchor=north](ST)at([yshift=-0.5cm]T.south){\ifdefvoid{\@subtitle}{}{\@subtitle}};
%%书脊出版社名称%%
\node[spine subtitle,anchor=south](PN)at([yshift=0.5cm]B){\ifdefvoid{\@pressname}{}{\@pressname}};
%%书脊出版社logo%%
\node[spine subtitle,anchor=south](PL)at([yshift=0.25cm]PN.north){\ifdefvoid{\@presslogo}{}{\includegraphics[width=0.8cm]{\@presslogo}}};
\end{tikzpicture}
\end{titlepage}}
\makeatother
\usepackage{zhlipsum,lipsum}%乱数假文(测试文字)
\graphicspath{{./figure/}{./figures/}{./image/}{./images/}{./graphics/}{./graphic/}{./pictures/}{./picture/}}%提供多种图片路径

%%预置索引栏目%%
\makeindex[name=noun,title=名词索引,columns=3,columnsep=2.0em,intoc,options=-s indexstyle.ist]
\makeindex[name=video,title=视频索引,columns=2,columnsep=2.0em,options=-s indexstyle.ist]
%%设置英文索引命令%%
\newcommand{\indexvideo}[1]{\index[video]{#1}}
\newcommand{\indexnoun}[1]{\index[noun]{#1}}
%%设置中文索引命令%%
\newcommand{\zhindexvideo}[2]{\zhindex[video]{#1}{#2}}
\newcommand{\zhindexnoun}[2]{\zhindex[noun]{#1}{#2}}

%\externaldocument{Ch1}%跨文件交叉引用
\series{物理学系列教材}
\title{物理学}
\englishtitle{PHYSICS}
\subtitle{力学部分}
\englishsubtitle{MECHANICS}
\author{作者}
\version{一}
\coverimage{coverimage.pdf}
\backimage{coverimage.pdf}
\pressname{出版社}
\presslogo{gj.png}



%\setmainfont{Times New Roman}

%%%%%%%%%%%%%%%%%%%%%%%%%%%%%%%%%%%%%%%%%%%%%%%%%%%%%%%%%%%%%%%%%%%%%%%%%%%%%%%%%%%
\begin{document}
\fontsize{12.5pt}{0.8\baselineskip}\selectfont

\makecover


\maketitle



\frontmatter



\chaptersubtitle{第一版}
\chapter*{序言}
\subsubsection{关于作者}\footnote{英明睿智的作者}
\lipsum
\subsubsection{关于本书}


\chaptersubtitle{致读者}
\chapter*{写在前面}
\lipsum
Aachen\indexvideo{Aachen} aardvark\indexvideo{aardvark} aargh\indexvideo{aargh} Aarhus\indexvideo{Aarhus} Aaron\indexvideo{Aaron} Aaronvitch\indexvideo{Aaronvitch} Ababa\indexvideo{Ababa} aback\indexvideo{aback} abacus\indexvideo{abacus} abaft\indexvideo{abaft} abalone\indexvideo{abalone} abandon\indexvideo{abandon} abandoner\indexvideo{abandoner} abase\indexvideo{abase} abaser\indexvideo{abaser} abash\indexvideo{abash} abashed\indexvideo{abashed} abate\indexvideo{abate} abated\indexvideo{abated} abater\indexvideo{abater} abattoir\indexvideo{abattoir} Abba\indexvideo{Abba} abbess\indexvideo{abbess} abbey\indexvideo{abbey} abbot\indexvideo{abbot} Abbott\indexvideo{Abbott} abbreviate\indexvideo{abbreviate} abbreviated\indexvideo{abbreviated} abbreviation\indexvideo{abbreviation} abbé\indexvideo{abbé} ABC\indexvideo{ABC} abdicate\indexvideo{abdicate} abdication\indexvideo{abdication} abdomen\indexvideo{abdomen} abdominal\indexvideo{abdominal} abduct\indexvideo{abduct} abductee\indexvideo{abductee} abduction\indexvideo{abduction} abductor\indexvideo{abductor} Abdul\indexvideo{Abdul} Abe\indexvideo{Abe} abeam\indexvideo{abeam} Abel\indexvideo{Abel} Abelard\indexvideo{Abelard} Abelson\indexvideo{Abelson} Aberconwy\indexvideo{Aberconwy} Aberdeen\indexvideo{Aberdeen} Aberdeenshire\indexvideo{Aberdeenshire} Abernathy\indexvideo{Abernathy} aberrant\indexvideo{aberrant} aberration\indexvideo{aberration} Aberystwyth\indexvideo{Aberystwyth} abet\indexvideo{abet} abettor\indexvideo{abettor} abeyance\indexvideo{abeyance} 
Aachen\indexnoun{Aachen} aardvark\indexnoun{aardvark} aargh\indexnoun{aargh} Aarhus\indexnoun{Aarhus} Aaron\indexnoun{Aaron} Aaronvitch\indexnoun{Aaronvitch} Ababa\indexnoun{Ababa} aback\indexnoun{aback} abacus\indexnoun{abacus} abaft\indexnoun{abaft} abalone\indexnoun{abalone} abandon\indexnoun{abandon} abandoner\indexnoun{abandoner} abase\indexnoun{abase} abaser\indexnoun{abaser} abash\indexnoun{abash} abashed\indexnoun{abashed} abate\indexnoun{abate} abated\indexnoun{abated} abater\indexnoun{abater} abattoir\indexnoun{abattoir} Abba\indexnoun{Abba} abbess\indexnoun{abbess} abbey\indexnoun{abbey} abbot\indexnoun{abbot} Abbott\indexnoun{Abbott} abbreviate\indexnoun{abbreviate} abbreviated\indexnoun{abbreviated} abbreviation\indexnoun{abbreviation} abbé\indexnoun{abbé} ABC\indexnoun{ABC} abdicate\indexnoun{abdicate} abdication\indexnoun{abdication} abdomen\indexnoun{abdomen} abdominal\indexnoun{abdominal} abduct\indexnoun{abduct} abductee\indexnoun{abductee} abduction\indexnoun{abduction} abductor\indexnoun{abductor} Abdul\indexnoun{Abdul} Abe\indexnoun{Abe} abeam\indexnoun{abeam} Abel\indexnoun{Abel} Abelard\indexnoun{Abelard} Abelson\indexnoun{Abelson} Aberconwy\indexnoun{Aberconwy} Aberdeen\indexnoun{Aberdeen} Aberdeenshire\indexnoun{Aberdeenshire} Abernathy\indexnoun{Abernathy} aberrant\indexnoun{aberrant} aberration\indexnoun{aberration} Aberystwyth\indexnoun{Aberystwyth} abet\indexnoun{abet} abettor\indexnoun{abettor} abeyance\indexnoun{abeyance} abeyant\indexnoun{abeyant} abhor\indexnoun{abhor} abhorrence\indexnoun{abhorrence} abhorrent\indexnoun{abhorrent} abhorrer\indexnoun{abhorrer} abidance\indexnoun{abidance} abide\indexnoun{abide} abider\indexnoun{abider} Abidjan\indexnoun{Abidjan} Abigail\indexnoun{Abigail} Abilene\indexnoun{Abilene} ability\indexnoun{ability} abiogenesis\indexnoun{abiogenesis} abiogenic\indexnoun{abiogenic} abiotic\indexnoun{abiotic} abject\indexnoun{abject} abjection\indexnoun{abjection} abjectness\indexnoun{abjectness} abjuration\indexnoun{abjuration} abjure\indexnoun{abjure} ablate\indexnoun{ablate} ablation\indexnoun{ablation} ablaze\indexnoun{ablaze} able-bodied\indexnoun{able-bodied} able\indexnoun{able} abler\indexnoun{abler} abloom\indexnoun{abloom} ablution\indexnoun{ablution} abnegate\indexnoun{abnegate} abnegation\indexnoun{abnegation} abnormal\indexnoun{abnormal} abnormality\indexnoun{abnormality} aboard\indexnoun{aboard} abode\indexnoun{abode} abolish\indexnoun{abolish} abolition\indexnoun{abolition} abolitionism\indexnoun{abolitionism} abolitionist\indexnoun{abolitionist} abominable\indexnoun{abominable} abominate\indexnoun{abominate} abomination\indexnoun{abomination} aboriginal\indexnoun{aboriginal} Aboriginals\indexnoun{Aboriginals} aborigine\indexnoun{aborigine} abort\indexnoun{abort} aborter\indexnoun{aborter} abortion\indexnoun{abortion} abortionist\indexnoun{abortionist} abortive\indexnoun{abortive} abound\indexnoun{abound} about\indexnoun{about} above\indexnoun{above} aboveboard\indexnoun{aboveboard} aboveground\indexnoun{aboveground} abracadabra\indexnoun{abracadabra} abrade\indexnoun{abrade} abrader\indexnoun{abrader} Abraham\indexnoun{Abraham} Abram\indexnoun{Abram} abrasion\indexnoun{abrasion} abrasive\indexnoun{abrasive} abrasiveness\indexnoun{abrasiveness} abreaction\indexnoun{abreaction} abreast\indexnoun{abreast} abridge\indexnoun{abridge} abridged\indexnoun{abridged} abridger\indexnoun{abridger} abroad\indexnoun{abroad} abrogate\indexnoun{abrogate} abrogation\indexnoun{abrogation} abrogator\indexnoun{abrogator} abrupt\indexnoun{abrupt} abruptness\indexnoun{abruptness} abs\indexnoun{abs} abscess\indexnoun{abscess} abscissa\indexnoun{abscissa} abscissae\indexnoun{abscissae} abscission\indexnoun{abscission} abscond\indexnoun{abscond} abseil\indexnoun{abseil} abseiler\indexnoun{abseiler} absence\indexnoun{absence} absent-minded\indexnoun{absent-minded} absent-mindedness\indexnoun{absent-mindedness} absent\indexnoun{absent} absentee\indexnoun{absentee} absenteeism\indexnoun{absenteeism} absenter\indexnoun{absenter} absentia\indexnoun{absentia} absinthe\indexnoun{absinthe} absolute\indexnoun{absolute} absoluteness\indexnoun{absoluteness} absolution\indexnoun{absolution} absolutism\indexnoun{absolutism} absolve\indexnoun{absolve} absolver\indexnoun{absolver} absorb\indexnoun{absorb} absorbed\indexnoun{absorbed} absorbency\indexnoun{absorbency} absorbent\indexnoun{absorbent} absorbs\indexnoun{absorbs} absorption\indexnoun{absorption} absorptivity\indexnoun{absorptivity} abstain\indexnoun{abstain} abstemious\indexnoun{abstemious} abstemiousness\indexnoun{abstemiousness} abstention\indexnoun{abstention} abstinence\indexnoun{abstinence} abstinent\indexnoun{abstinent} abstract\indexnoun{abstract} abstracted\indexnoun{abstracted} abstractedness\indexnoun{abstractedness} abstracter\indexnoun{abstracter} abstraction\indexnoun{abstraction} abstractionism\indexnoun{abstractionism} abstractionist\indexnoun{abstractionist} abstractness\indexnoun{abstractness} abstractor\indexnoun{abstractor} abstruse\indexnoun{abstruse} abstruseness\indexnoun{abstruseness} absurd\indexnoun{absurd} absurdity\indexnoun{absurdity} absurdness\indexnoun{absurdness} Abu\indexnoun{Abu} Abuja\indexnoun{Abuja} abundance\indexnoun{abundance} abundant\indexnoun{abundant} abusable\indexnoun{abusable} abuse\indexnoun{abuse} abuser\indexnoun{abuser} abusive\indexnoun{abusive} abusiveness\indexnoun{abusiveness} abut\indexnoun{abut} abuzz\indexnoun{abuzz} abysmal\indexnoun{abysmal} abyss\indexnoun{abyss} abyssal\indexnoun{abyssal} Abyssinia\indexnoun{Abyssinia} Abyssinian\indexnoun{Abyssinian} Ac\indexnoun{Ac} AC\indexnoun{AC} acacia\indexnoun{acacia} academe\indexnoun{academe} academia\indexnoun{academia} academic\indexnoun{academic} academician\indexnoun{academician} academicianship\indexnoun{academicianship} academy\indexnoun{academy} Acadia\indexnoun{Acadia} acanthus\indexnoun{acanthus} Acapulco\indexnoun{Acapulco} ACAS\indexnoun{ACAS} acc.\indexnoun{acc.} accede\indexnoun{accede} accelerate\indexnoun{accelerate} acceleration\indexnoun{acceleration} accelerator\indexnoun{accelerator} accelerometer\indexnoun{accelerometer} accent\indexnoun{accent} accented\indexnoun{accented} accentual\indexnoun{accentual} accentuate\indexnoun{accentuate} accentuation\indexnoun{accentuation} accept\indexnoun{accept} acceptability\indexnoun{acceptability} acceptable\indexnoun{acceptable} acceptableness\indexnoun{acceptableness} acceptably\indexnoun{acceptably} acceptance\indexnoun{acceptance} acceptant\indexnoun{acceptant} accepted\indexnoun{accepted} acceptingness\indexnoun{acceptingness} acceptor\indexnoun{acceptor} access\indexnoun{access} accessibility\indexnoun{accessibility} accessible\indexnoun{accessible} accessibly\indexnoun{accessibly} accession\indexnoun{accession} accessors\indexnoun{accessors} accessory\indexnoun{accessory} accidence\indexnoun{accidence} accident-prone\indexnoun{accident-prone} accident\indexnoun{accident} accidental\indexnoun{accidental} acclaim\indexnoun{acclaim} acclaimer\indexnoun{acclaimer} acclamation\indexnoun{acclamation} acclimate\indexnoun{acclimate} acclimation\indexnoun{acclimation} acclimatisation\indexnoun{acclimatisation} acclimatise\indexnoun{acclimatise} acclimatization\indexnoun{acclimatization} acclimatize\indexnoun{acclimatize} acclivity\indexnoun{acclivity} accolade\indexnoun{accolade} accommodate\indexnoun{accommodate} accommodating\indexnoun{accommodating} accommodation\indexnoun{accommodation} accommodative\indexnoun{accommodative} accompanied\indexnoun{accompanied} accompanier\indexnoun{accompanier} accompany\indexnoun{accompany} accomplice\indexnoun{accomplice} accomplish\indexnoun{accomplish} accomplished\indexnoun{accomplished} accord\indexnoun{accord} accordance\indexnoun{accordance} accordant\indexnoun{accordant} accordion\indexnoun{accordion} accordionist\indexnoun{accordionist} accost\indexnoun{accost} account\indexnoun{account} accountability's\indexnoun{accountability's} accountability\indexnoun{accountability} accountable\indexnoun{accountable} accountably\indexnoun{accountably} accountancy\indexnoun{accountancy} accountant\indexnoun{accountant} accounted\indexnoun{accounted} accounting\indexnoun{accounting} accoutre\indexnoun{accoutre} Accra\indexnoun{Accra} accredit\indexnoun{accredit} accreditation\indexnoun{accreditation} accredited\indexnoun{accredited} accreted\indexnoun{accreted} accretion\indexnoun{accretion} accretive\indexnoun{accretive} accrual\indexnoun{accrual} accrue\indexnoun{accrue} acct\indexnoun{acct} acculturate\indexnoun{acculturate} acculturation\indexnoun{acculturation} accumulate\indexnoun{accumulate} accumulation\indexnoun{accumulation} accumulative\indexnoun{accumulative} accumulator\indexnoun{accumulator} accuracy\indexnoun{accuracy} accurate\indexnoun{accurate} accurately\indexnoun{accurately} accurateness\indexnoun{accurateness} accursed\indexnoun{accursed} accursedness\indexnoun{accursedness} accusal\indexnoun{accusal} accusation\indexnoun{accusation} accusative\indexnoun{accusative} accusatory\indexnoun{accusatory} accuse\indexnoun{accuse} accused\indexnoun{accused} accustom\indexnoun{accustom} accustomed\indexnoun{accustomed} accustomedness\indexnoun{accustomedness} ace\indexnoun{ace} acellular\indexnoun{acellular} acer\indexnoun{acer} acerbate\indexnoun{acerbate} acerbic\indexnoun{acerbic} acerbity\indexnoun{acerbity} acetaminophen\indexnoun{acetaminophen} acetate\indexnoun{acetate} acetic\indexnoun{acetic} acetogenic\indexnoun{acetogenic} acetone\indexnoun{acetone} acetylcholine\indexnoun{acetylcholine} acetylene\indexnoun{acetylene} Achaean\indexnoun{Achaean} ache\indexnoun{ache} ached\indexnoun{ached} achene\indexnoun{achene} aches\indexnoun{aches} achievable\indexnoun{achievable} achieve\indexnoun{achieve} achieved\indexnoun{achieved} achievement's\indexnoun{achievement's} achievement\indexnoun{achievement} achievements\indexnoun{achievements} achiever\indexnoun{achiever} achieves\indexnoun{achieves} achieving\indexnoun{achieving} Achilles\indexnoun{Achilles} aching\indexnoun{aching} achromatic\indexnoun{achromatic} achy\indexnoun{achy} acid\indexnoun{acid} acidification\indexnoun{acidification} acidify\indexnoun{acidify} acidity\indexnoun{acidity} acidophil\indexnoun{acidophil} acidophiles\indexnoun{acidophiles} acidoses\indexnoun{acidoses} acidosis\indexnoun{acidosis} acidulous\indexnoun{acidulous} Ackerman\indexnoun{Ackerman} acknowledge\indexnoun{acknowledge} acknowledgeable\indexnoun{acknowledgeable} acknowledged\indexnoun{acknowledged} ACM\indexnoun{ACM} acme\indexnoun{acme} acne\indexnoun{acne} acolyte\indexnoun{acolyte} aconite\indexnoun{aconite} acorn\indexnoun{acorn} acoustic\indexnoun{acoustic} acoustical\indexnoun{acoustical} acoustician\indexnoun{acoustician} acoustics\indexnoun{acoustics} acquaint\indexnoun{acquaint} acquaintance\indexnoun{acquaintance} acquaintanceship\indexnoun{acquaintanceship} acquainted\indexnoun{acquainted} acquiesce\indexnoun{acquiesce} acquiescence\indexnoun{acquiescence} acquiescent\indexnoun{acquiescent} acquirable\indexnoun{acquirable} acquire\indexnoun{acquire} acquirement\indexnoun{acquirement} acquisition\indexnoun{acquisition} acquisitions\indexnoun{acquisitions} acquisitive\indexnoun{acquisitive} acquisitiveness\indexnoun{acquisitiveness} acquit\indexnoun{acquit} acquittal\indexnoun{acquittal} acquittance\indexnoun{acquittance} acquitter\indexnoun{acquitter} acre\indexnoun{acre} acreage\indexnoun{acreage} acrid\indexnoun{acrid} acridity\indexnoun{acridity} acridness\indexnoun{acridness} acrimonious\indexnoun{acrimonious} acrimoniousness\indexnoun{acrimoniousness} acrimony\indexnoun{acrimony} acrobat\indexnoun{acrobat} acrobatic\indexnoun{acrobatic} acrobatics\indexnoun{acrobatics} acronym\indexnoun{acronym} acrophobia\indexnoun{acrophobia} acropolis\indexnoun{acropolis} across\indexnoun{across} acrostic\indexnoun{acrostic} acrylate\indexnoun{acrylate} acrylic\indexnoun{acrylic} act's\indexnoun{act's} act\indexnoun{act} actinic\indexnoun{actinic} actinide\indexnoun{actinide} actinium\indexnoun{actinium} actinometer\indexnoun{actinometer} action\indexnoun{action} actionable\indexnoun{actionable} actioned\indexnoun{actioned} actioning\indexnoun{actioning} activate\indexnoun{activate} activated\indexnoun{activated} activating\indexnoun{activating} activation\indexnoun{activation} activator\indexnoun{activator} active\indexnoun{active} actively\indexnoun{actively} activeness\indexnoun{activeness} activism\indexnoun{activism} activity\indexnoun{activity} Acton\indexnoun{Acton} actor\indexnoun{actor} actress\indexnoun{actress} actual\indexnoun{actual} actuality\indexnoun{actuality} actuarial\indexnoun{actuarial} actuary\indexnoun{actuary} actuate\indexnoun{actuate} actuation\indexnoun{actuation} actuator\indexnoun{actuator} acuity\indexnoun{acuity} acumen\indexnoun{acumen} acupressure\indexnoun{acupressure} acupuncture\indexnoun{acupuncture} acute\indexnoun{acute} acuteness\indexnoun{acuteness} acyclic\indexnoun{acyclic} acyclovir\indexnoun{acyclovir} ad's\indexnoun{ad's} ad\indexnoun{ad} adage\indexnoun{adage} adagio\indexnoun{adagio} Adair\indexnoun{Adair} Adam\indexnoun{Adam} adamant\indexnoun{adamant} Adamski\indexnoun{Adamski} Adamson\indexnoun{Adamson} adapt\indexnoun{adapt} adaptability\indexnoun{adaptability} adaptation\indexnoun{adaptation} adapted\indexnoun{adapted} adaptive\indexnoun{adaptive} adaptivity\indexnoun{adaptivity} adaptor\indexnoun{adaptor} ADC\indexnoun{ADC} add-on\indexnoun{add-on} add\indexnoun{add} addend\indexnoun{addend} addenda\indexnoun{addenda} addendum\indexnoun{addendum} addict\indexnoun{addict} addiction\indexnoun{addiction} Addis\indexnoun{Addis} Addison\indexnoun{Addison} addition\indexnoun{addition} additional\indexnoun{additional} additive\indexnoun{additive} additivity\indexnoun{additivity} addle\indexnoun{addle} addorsed\indexnoun{addorsed} address\indexnoun{address} addressed\indexnoun{addressed} addressee\indexnoun{addressee} addresses\indexnoun{addresses} addressing\indexnoun{addressing} adduce\indexnoun{adduce} adducer\indexnoun{adducer} adduct\indexnoun{adduct} adduction\indexnoun{adduction} adductor\indexnoun{adductor} Adelaide\indexnoun{Adelaide} Adele\indexnoun{Adele} Aden\indexnoun{Aden} Adenauer\indexnoun{Adenauer} adenine\indexnoun{adenine} adenoid\indexnoun{adenoid} adenoidal\indexnoun{adenoidal} adenoma\indexnoun{adenoma} adenomata\indexnoun{adenomata} adenoviral\indexnoun{adenoviral} adenovirus\indexnoun{adenovirus} adept\indexnoun{adept} adeptness\indexnoun{adeptness} adequacy\indexnoun{adequacy} adequate\indexnoun{adequate} adequateness\indexnoun{adequateness} adhere\indexnoun{adhere} adherence\indexnoun{adherence} adherent\indexnoun{adherent} adhesion\indexnoun{adhesion} adhesive\indexnoun{adhesive} adhesiveness\indexnoun{adhesiveness} adiabatic\indexnoun{adiabatic} adieu\indexnoun{adieu} adipose\indexnoun{adipose} Adirondack\indexnoun{Adirondack} adiós\indexnoun{adiós} adjacency\indexnoun{adjacency} adjacent\indexnoun{adjacent} adjectival\indexnoun{adjectival} adjective\indexnoun{adjective} adjoin\indexnoun{adjoin} adjourn\indexnoun{adjourn} adjudge\indexnoun{adjudge} adjudicate\indexnoun{adjudicate} adjudication\indexnoun{adjudication} adjudicator\indexnoun{adjudicator} adjunct\indexnoun{adjunct} adjuration\indexnoun{adjuration} adjure\indexnoun{adjure} adjust\indexnoun{adjust} adjusted\indexnoun{adjusted} adjustor's\indexnoun{adjustor's} adjusts\indexnoun{adjusts} adjutant\indexnoun{adjutant} Adkins\indexnoun{Adkins} Adler\indexnoun{Adler} adman\indexnoun{adman} admen\indexnoun{admen} admin\indexnoun{admin} adminicle\indexnoun{adminicle} adminicular\indexnoun{adminicular} administer\indexnoun{administer} administrable\indexnoun{administrable} administrate\indexnoun{administrate} administration\indexnoun{administration} administrator\indexnoun{administrator} administratrix\indexnoun{administratrix} admirable\indexnoun{admirable} admiral\indexnoun{admiral} admiralty\indexnoun{admiralty} admiration\indexnoun{admiration} admire\indexnoun{admire} admissibility\indexnoun{admissibility} admissible\indexnoun{admissible} admission\indexnoun{admission} admit\indexnoun{admit} admittance\indexnoun{admittance} admitted\indexnoun{admitted} admixture\indexnoun{admixture} admonish\indexnoun{admonish} admonisher\indexnoun{admonisher} admonition\indexnoun{admonition} admonitory\indexnoun{admonitory} ado\indexnoun{ado} adobe\indexnoun{adobe} adolescence's\indexnoun{adolescence's} adolescence\indexnoun{adolescence} adolescent\indexnoun{adolescent} Adolph\indexnoun{Adolph} Adonis\indexnoun{Adonis} adopt\indexnoun{adopt} adopted\indexnoun{adopted} adoption\indexnoun{adoption} adopts\indexnoun{adopts} adorable\indexnoun{adorable} adorableness\indexnoun{adorableness} adoration\indexnoun{adoration} adore\indexnoun{adore} adorn\indexnoun{adorn} adorned\indexnoun{adorned} adrenal\indexnoun{adrenal} adrenalin\indexnoun{adrenalin} adrenaline\indexnoun{adrenaline} Adrian\indexnoun{Adrian} Adriatic\indexnoun{Adriatic} Adrienne\indexnoun{Adrienne} adrift\indexnoun{adrift} adroit\indexnoun{adroit} adroitness\indexnoun{adroitness} ads\indexnoun{ads} ADSL\indexnoun{ADSL} adsorb\indexnoun{adsorb} adsorbate\indexnoun{adsorbate} adsorbent\indexnoun{adsorbent} adsorption\indexnoun{adsorption} ADte\indexnoun{ADte} adulate\indexnoun{adulate} adulation\indexnoun{adulation} adulator\indexnoun{adulator} adult\indexnoun{adult} adulterant\indexnoun{adulterant} adulterate\indexnoun{adulterate} adulterated\indexnoun{adulterated} adulteration\indexnoun{adulteration} adulterer\indexnoun{adulterer} adulteress\indexnoun{adulteress} adulterous\indexnoun{adulterous} adultery\indexnoun{adultery} adulthood\indexnoun{adulthood} adumbrate\indexnoun{adumbrate} adumbration\indexnoun{adumbration} advance\indexnoun{advance} advantage\indexnoun{advantage} advantageous\indexnoun{advantageous} advantageousness's\indexnoun{advantageousness's} advantageousness\indexnoun{advantageousness} advent\indexnoun{advent} Adventist's\indexnoun{Adventist's} adventist\indexnoun{adventist} adventitious\indexnoun{adventitious} adventure's\indexnoun{adventure's} adventure\indexnoun{adventure} adventures\indexnoun{adventures} adventuresome\indexnoun{adventuresome} adventuress\indexnoun{adventuress} adventurism\indexnoun{adventurism} adventurous\indexnoun{adventurous} adventurously\indexnoun{adventurously} adventurousness\indexnoun{adventurousness} adverb\indexnoun{adverb} adverbial\indexnoun{adverbial} adversarial\indexnoun{adversarial} adversary\indexnoun{adversary} adverse\indexnoun{adverse} adverseness\indexnoun{adverseness} adversity\indexnoun{adversity} advert\indexnoun{advert} advertise\indexnoun{advertise} advertised\indexnoun{advertised} advertising\indexnoun{advertising} advice\indexnoun{advice} advisabilities\indexnoun{advisabilities} advisability's\indexnoun{advisability's} advisability\indexnoun{advisability} advisable\indexnoun{advisable} advise\indexnoun{advise} advisedly\indexnoun{advisedly} advisee\indexnoun{advisee} advisor\indexnoun{advisor} advisory\indexnoun{advisory} advocacy\indexnoun{advocacy} advocate\indexnoun{advocate} adware\indexnoun{adware} adze\indexnoun{adze} Aegean\indexnoun{Aegean} aegis\indexnoun{aegis} aegrotat\indexnoun{aegrotat} Aeneas\indexnoun{Aeneas} Aeneid\indexnoun{Aeneid} aeolian\indexnoun{aeolian} Aeolus\indexnoun{Aeolus} aeon\indexnoun{aeon} aerate\indexnoun{aerate} aeration\indexnoun{aeration} aerator\indexnoun{aerator} aerial\indexnoun{aerial} aerialist\indexnoun{aerialist} aerie\indexnoun{aerie} aero\indexnoun{aero} aero-engine\indexnoun{aero-engine} aeroacoustic\indexnoun{aeroacoustic} aerobatic\indexnoun{aerobatic} aerobic\indexnoun{aerobic} aerodrome\indexnoun{aerodrome} aerodynamic\indexnoun{aerodynamic} aerodynamics\indexnoun{aerodynamics} aeroelastic\indexnoun{aeroelastic} aeroelasticity\indexnoun{aeroelasticity} aerofoil\indexnoun{aerofoil} aerogel\indexnoun{aerogel} aerogramme\indexnoun{aerogramme} aerolite\indexnoun{aerolite} aeromagnetic\indexnoun{aeromagnetic} aeromedical\indexnoun{aeromedical} aeromodeller\indexnoun{aeromodeller} aeromodelling\indexnoun{aeromodelling} aeronautic\indexnoun{aeronautic} aeronautical\indexnoun{aeronautical} aeronautics\indexnoun{aeronautics} aerophone\indexnoun{aerophone} aeroplane\indexnoun{aeroplane} aeroponic\indexnoun{aeroponic} aeroponically\indexnoun{aeroponically} aerosol\indexnoun{aerosol} aerospace\indexnoun{aerospace} Aeschylus\indexnoun{Aeschylus} Aesculapius\indexnoun{Aesculapius} Aesop\indexnoun{Aesop} aesthete\indexnoun{aesthete} aesthetic\indexnoun{aesthetic} aestheticism\indexnoun{aestheticism} aestival\indexnoun{aestival} aestivate\indexnoun{aestivate} aether\indexnoun{aether} aetiology\indexnoun{aetiology} AFAIK\indexnoun{AFAIK} afar\indexnoun{afar} AFC\indexnoun{AFC} affability\indexnoun{affability} affable\indexnoun{affable} affair\indexnoun{affair} affect\indexnoun{affect} affectation\indexnoun{affectation} affected\indexnoun{affected} affectedly\indexnoun{affectedly} affecter\indexnoun{affecter} affecting\indexnoun{affecting} affection\indexnoun{affection} affectionate\indexnoun{affectionate} affectionately\indexnoun{affectionately} affective\indexnoun{affective} affects\indexnoun{affects} afferent\indexnoun{afferent} affiance\indexnoun{affiance} affidavit\indexnoun{affidavit} affiliate\indexnoun{affiliate} affiliated\indexnoun{affiliated} affiliation\indexnoun{affiliation} affine\indexnoun{affine} affinity\indexnoun{affinity} affirm\indexnoun{affirm} affirmation\indexnoun{affirmation} affirmed\indexnoun{affirmed} affirms\indexnoun{affirms} affix\indexnoun{affix} afflatus\indexnoun{afflatus} afflict\indexnoun{afflict} affliction\indexnoun{affliction} affluence\indexnoun{affluence} affluent\indexnoun{affluent} afford\indexnoun{afford} affordable\indexnoun{affordable} afforest\indexnoun{afforest} afforestation\indexnoun{afforestation} affray\indexnoun{affray} affricate\indexnoun{affricate} affrication\indexnoun{affrication} affricative\indexnoun{affricative} affright\indexnoun{affright} affront\indexnoun{affront} Afghan\indexnoun{Afghan} Afghani\indexnoun{Afghani} Afghanistan\indexnoun{Afghanistan} aficionado\indexnoun{aficionado} afield\indexnoun{afield} afire\indexnoun{afire} aflame\indexnoun{aflame}

\nomenclature{$\bm a$}{加速度}
\nomenclature{$\bm a^*$}{平均加速度}
\nomenclature{$\bm a_a$}{绝对加速度}
\nomenclature{$\bm a_C$}{质心加速度}
\nomenclature{$\bm a_e$}{牵连加速度}
\nomenclature{$\bm a_n$}{法向加速度}
\nomenclature{$\bm a_t$}{切向加速度}
\nomenclature{$A$}{面积,自由振动振幅}
\nomenclature{$\bm b$}{副法线基矢量}
\nomenclature{$\rm d$}{微分符号}
\nomenclature{$d$}{直径}
\nomenclature{$C$}{质心}
\nomenclature{$\bm e$}{单位矢量}
\nomenclature{$E$}{总机械能}
\nomenclature{$f$}{动摩擦因数,频率}
\nomenclature{$f_s$}{静摩擦因数}
\nomenclature{$\bm F$}{力}
\nomenclature{$\bm F_I$}{达朗贝尔惯性力}
\nomenclature{$\bm F_{IC}$}{科氏惯性力}
\nomenclature{$\bm F_{Ie}$}{牵连惯性力}
\nomenclature{$\bm F_N$}{法向约束力}
\nomenclature{$\bm F_P$}{主动力}
\nomenclature{$\bm F_R$}{力系的主矢,全反力}
\nomenclature{$\bm F_S$}{桁架中杆的内力}
\nomenclature{$\bm F_T$}{绳的张力}
\nomenclature{$\bm g$}{重力加速度}
\nomenclature{$h$}{高度}
\nomenclature{$\bm i$}{$x$~轴的基矢量}
\nomenclature{$\bm I$}{冲量}
\nomenclature{$\bm j$}{$y$~轴的基矢量}
\nomenclature{$\bm J_z$}{刚体对~$z$~轴的转动惯量}
\nomenclature{$k$}{弹簧刚度系数}
\nomenclature{$\bm k$}{$z$~轴的基矢量}
\nomenclature{$l$}{长度}
\nomenclature{$\bm L_C$}{刚体对质心的动量矩}
\nomenclature{$\bm L_O$}{刚体对点~$O$~的动量矩}
\nomenclature{$m$}{质量}
\nomenclature{$\bm M$}{力偶矩}
\nomenclature{$\bm M_I$}{惯性力系的主矩}
\nomenclature{$\bm M_O(F)$}{$F$~对点~$O$~轴的矩}
\nomenclature{$M_z(\bm F)$}{$F$~对~$z$~轴的矩}
\nomenclature{$n$}{质点数目}
\nomenclature{$\bm n$}{法线基矢量}
\nomenclature{$\bm p$}{动量}
\nomenclature{$q$}{载荷集度}
\nomenclature{$r$}{半径}
\nomenclature{$\bm r$}{矢径}
\nomenclature{$\bm r_C$}{质心的矢径}
\nomenclature{$\bm r_M$}{点~$M$~的矢径}
\nomenclature{$R$}{半径}
\nomenclature{$s$}{弧坐标}
\nomenclature{$t$}{时间}
\nomenclature{$T$}{动能,周期}
\nomenclature{$T_r$}{相对运动动能}
\nomenclature{$U$}{势力函数}
\nomenclature{$\bm v$}{速度}
\nomenclature{$\bm v^*$}{平均速度}
\nomenclature{$\bm v_a$}{绝对速度}
\nomenclature{$\bm v_C$}{质心速度}
\nomenclature{$\bm v_e$}{牵连速度}
\nomenclature{$\bm v_r$}{相对速度}
\nomenclature{$V$}{势能,体积}
\nomenclature{$W$}{力的功}
\nomenclature{$\bm W$}{重力}
\nomenclature{$x,y,z$}{直角坐标}
\nomenclature{$\bm\alpha$}{角加速度}
\nomenclature{$\alpha,\beta,\gamma$}{方位角}
\nomenclature{$\theta,\phi$}{常用角}
\nomenclature{$\delta$}{滚阻系数}
\nomenclature{$\updelta$}{变分符号}
\nomenclature{$\Delta$}{增量符号}
\nomenclature{$\kappa$}{曲率}
\nomenclature{$\kappa^*$}{平均曲率}
\nomenclature{$\lambda$}{弹簧的变形}
\nomenclature{$\rho$}{密度,曲率半径}
\nomenclature{$\bm\rho$}{相对矢径}
\nomenclature{$\bm\rho_C$}{质心的相对矢径}
\nomenclature{$\bm\tau$}{切线基矢量}
\nomenclature{$\phi_m$}{摩擦角}
\nomenclature{$\omega_0$}{固有角频率}
\nomenclature{$\bm\omega$}{角速度}

\printnomenclature

\tableofcontents

\listoffigures

\mainmatter
\tcbstartrecording



\part{绪论}

\chaptersaying{\lipsum[1][1]}
\chapter{物理学的研究内容与方法}
\lipsum


\chaptersaying{\lipsum[1][2]}
\chapter{物理学与其他学科}
\lipsum




\part{运动学}

\chaptersaying{\lipsum[1][3]}
\chapter{直线运动}
\begin{Block*}[章节引入]
\lipsum[1-9]
\end{Block*}
\lipsum[1]

\section{机械运动}

\ExerciseTitle{基础巩固}

\begin{multicols}{2}

\begin{QsNum}
\item \lipsum[1][1]
\item \lipsum[1][1]
\item \lipsum[1][1]
\item \lipsum[1][1]
\item \lipsum[1][1]
\item \lipsum[1][1]
\item \lipsum[1][1]
\item \lipsum[1][1]
\item \lipsum[1][1]
\item \lipsum[1][1]
\item \lipsum[1][1]
\item \lipsum[1][1]
\end{QsNum}

\end{multicols}

\ExerciseTitle{技能提高}

\begin{multicols}{2}

\begin{QsNum}
\item \lipsum[1][1]
\item \lipsum[1][1]
\item \lipsum[1][1]
\item \lipsum[1][1]
\item \lipsum[1][1]
\item \lipsum[1][1]
\item \lipsum[1][1]
\item \lipsum[1][1]
\item \lipsum[1][1]
\item \lipsum[1][1]
\item \lipsum[1][1]
\item \lipsum[1][1]
\end{QsNum}

\end{multicols}

\ExerciseTitle{能力培养}

\begin{multicols}{2}

\begin{QsNum}
\item \lipsum[1][1]
\item \lipsum[1][1]
\item \lipsum[1][1]
\item \lipsum[1][1]
\item \lipsum[1][1]
\item \lipsum[1][1]
\item \lipsum[1][1]
\item \lipsum[1][1]
\item \lipsum[1][1]
\item \lipsum[1][1]
\item \lipsum[1][1]
\item \lipsum[1][1]
\item \lipsum[1][1]
\item \lipsum[1][1]
\item \lipsum[1][1]
\item \lipsum[1][1]
\item \lipsum[1][1]
\item \lipsum[1][1]
\end{QsNum}

\end{multicols}


\clearpage
\subsection{小节}
\subsection*{小节}



\Example{\lipsum[1]}
\Answer{\lipsum[2]}

\begin{figure}[H]
\includegraphics[width=5.0cm]{PLUS.pdf}
\caption{LOGO}
\end{figure}





\section{位置与位移}

\begin{Point}
\lipsum[2]
\end{Point}

\begin{Case*}
\item \lipsum[1][3]
\item \lipsum[1][3]
\item \lipsum[1][3]
\item \lipsum[1][3]
\end{Case*}


\subsection{小节}
\subsection*{小节}

\Answer{\lipsum[2][1]}


\begin{Link}
\zhlipsum[1]
\end{Link}


\clearpage
\begin{Thinking}
\begin{QsNum}
\item \lipsum[1][3]
\item \lipsum[1][3]
\item \lipsum[1][3]
\item \lipsum[1][3]
\end{QsNum}
\tcblower
\lipsum[1]
\end{Thinking}


\section{速度和加速度}

\begin{Point}
\lipsum[2]
\end{Point}

\begin{Case}
\item \lipsum[1][3]
\item \lipsum[1][3]
\item \lipsum[1][3]
\item \lipsum[1][3]
\end{Case}

\begin{Link}
\zhlipsum[1]
\end{Link}


\usetikzlibrary{shapes.callouts}
\begin{tikzpicture}
\node[draw=black,fill=white,rectangle callout,minimum height=4ex,inner xsep=2.5mm,rounded corners,anchor=pointer](S)at(5,0){Look!};
\end{tikzpicture}




\section{匀变速直线运动}

\begin{Point}
\lipsum[2]
\end{Point}

\begin{Case}
\item \lipsum[1][3]
\item \lipsum[1][3]
\item \lipsum[1][3]
\item \lipsum[1][3]
\end{Case}

\begin{Link}
\zhlipsum[1]
\end{Link}


\section{匀变速直线运动规律的应用}

\begin{Point}
\lipsum[2]
\end{Point}

\begin{Case}
\item \lipsum[1][3]
\item \lipsum[1][3]
\item \lipsum[1][3]
\item \lipsum[1][3]
\end{Case}

\begin{Link}
\zhlipsum[1]
\end{Link}




\section{自由落体和竖直上抛}

\begin{Point}
\lipsum[2]
\end{Point}

\begin{Case}
\item \lipsum[1][3]
\item \lipsum[1][3]
\item \lipsum[1][3]
\item \lipsum[1][3]
\end{Case}

\begin{Link}
\zhlipsum[1]
\end{Link}




\section{回顾与总结}

\begin{Point}
\lipsum[2]
\end{Point}

\begin{Case*}
\item \lipsum[1][3]
\item \lipsum[1][3]
\item \lipsum[1][3]
\item \lipsum[1][3]
\end{Case*}

\begin{Link}
\zhlipsum[1]
\end{Link}


\chaptersaying{\lipsum[1][4]}
\chapter{曲线运动}


\begin{Block}[章节引入]
\lipsum[1-3]
\end{Block}

\section{运动的合成与分解}
\begin{Point}
\lipsum[2]
\end{Point}

\begin{Case}
\item \lipsum[1][3]
\item \lipsum[1][3]
\item \lipsum[1][3]
\item \lipsum[1][3]
\end{Case}

\begin{Link}
\zhlipsum[1]
\end{Link}


\section{抛体运动}
\begin{Point}
\lipsum[2]
\end{Point}

\begin{Case}
\item \lipsum[1][3]
\item \lipsum[1][3]
\item \lipsum[1][3]
\item \lipsum[1][3]
\end{Case}

\begin{Link}
\zhlipsum[1]
\end{Link}




\section{圆周运动}
\begin{Point}
\lipsum[2]
\end{Point}

\begin{Case}
\item \lipsum[1][3]
\item \lipsum[1][3]
\item \lipsum[1][3]
\item \lipsum[1][3]
\end{Case}

\begin{Link}
\zhlipsum[1]
\end{Link}




\section{向心加速度}
\begin{Point}
\lipsum[2]
\end{Point}

\begin{Case}
\item \lipsum[1][3]
\item \lipsum[1][3]
\item \lipsum[1][3]
\item \lipsum[1][3]
\end{Case}

\begin{Link}
\zhlipsum[1]
\end{Link}



\section{圆周运动实例分析}
\begin{Point}
\lipsum[2]
\end{Point}

\begin{Case}
\item \lipsum[1][3]
\item \lipsum[1][3]
\item \lipsum[1][3]
\item \lipsum[1][3]
\end{Case}

\begin{Link}
\zhlipsum[1]
\end{Link}




\section{离心现象}

\begin{Point}
\lipsum[2]
\end{Point}

\begin{Case}
\item \lipsum[1][3]
\item \lipsum[1][3]
\item \lipsum[1][3]
\item \lipsum[1][3]
\end{Case}

\begin{Link}
\zhlipsum[1]
\end{Link}





\section{回顾与总结}
\begin{Point}
\lipsum[2]
\end{Point}

\begin{Case*}
\item \lipsum[1][3]
\item \lipsum[1][3]
\item \lipsum[1][3]
\item \lipsum[1][3]
\end{Case*}

\begin{Link}
\zhlipsum[1]
\end{Link}




\part{静力学}


\chaptersaying{\lipsum[1][5]}
\chapter{力和力矩}


\begin{Block}[章节引入]
\lipsum[1-3]
\end{Block}


\section{力的分类}

\begin{Point}
\lipsum[2]
\end{Point}

\begin{Case}
\item \lipsum[1][3]
\item \lipsum[1][3]
\item \lipsum[1][3]
\item \lipsum[1][3]
\end{Case}

\begin{Link}
\zhlipsum[1]
\end{Link}



\section{受力分析}

\begin{Point}
\lipsum[2]
\end{Point}

\begin{Case}
\item \lipsum[1][3]
\item \lipsum[1][3]
\item \lipsum[1][3]
\item \lipsum[1][3]
\end{Case}

\begin{Link}
\zhlipsum[1]
\end{Link}



\section{力的合成与分解}

\begin{Point}
\lipsum[2]
\end{Point}

\begin{Case}
\item \lipsum[1][3]
\item \lipsum[1][3]
\item \lipsum[1][3]
\item \lipsum[1][3]
\end{Case}

\begin{Link}
\zhlipsum[1]
\end{Link}



\section{牛顿第三定律}

\begin{Point}
\lipsum[2]
\end{Point}

\begin{Case}
\item \lipsum[1][3]
\item \lipsum[1][3]
\item \lipsum[1][3]
\item \lipsum[1][3]
\end{Case}

\begin{Link}
\zhlipsum[1]
\end{Link}







\section{认识力矩}

\begin{Point}
\lipsum[2]
\end{Point}

\begin{Case}
\item \lipsum[1][3]
\item \lipsum[1][3]
\item \lipsum[1][3]
\item \lipsum[1][3]
\end{Case}

\begin{Link}
\zhlipsum[1]
\end{Link}


\section{回顾与总结}

\begin{Point}
\lipsum[2]
\end{Point}

\begin{Case*}
\item \lipsum[1][3]
\item \lipsum[1][3]
\item \lipsum[1][3]
\item \lipsum[1][3]
\end{Case*}

\begin{Link}
\zhlipsum[1]
\end{Link}


\chaptersaying{\lipsum[1][6]}
\chapter{力与平衡}


\begin{Block}[章节引入]
\lipsum[1-3]
\end{Block}

\section{牛顿第一定律}

\begin{Point}
\lipsum[2]
\end{Point}

\begin{Case}
\item \lipsum[1][3]
\item \lipsum[1][3]
\item \lipsum[1][3]
\item \lipsum[1][3]
\end{Case}

\begin{Link}
\zhlipsum[1]
\end{Link}




\section{力的平衡}

\begin{Point}
\lipsum[2]
\end{Point}

\begin{Case}
\item \lipsum[1][3]
\item \lipsum[1][3]
\item \lipsum[1][3]
\item \lipsum[1][3]
\end{Case}

\begin{Link}
\zhlipsum[1]
\end{Link}



\section{力矩平衡}

\begin{Point}
\lipsum[2]
\end{Point}

\begin{Case}
\item \lipsum[1][3]
\item \lipsum[1][3]
\item \lipsum[1][3]
\item \lipsum[1][3]
\end{Case}

\begin{Link}
\zhlipsum[1]
\end{Link}



\section{平衡问题的求解}

\begin{Point}
\lipsum[2]
\end{Point}

\begin{Case}
\item \lipsum[1][3]
\item \lipsum[1][3]
\item \lipsum[1][3]
\item \lipsum[1][3]
\end{Case}

\begin{Link}
\zhlipsum[1]
\end{Link}



\section{回顾与总结}

\begin{Point}
\lipsum[2]
\end{Point}

\begin{Case*}
\item \lipsum[1][3]
\item \lipsum[1][3]
\item \lipsum[1][3]
\item \lipsum[1][3]
\end{Case*}

\begin{Link}
\zhlipsum[1]
\end{Link}






\part{动力学}

\chaptersaying{\lipsum[1][7]}
\chapter{力与运动}


\begin{Block}[章节引入]
\lipsum[1-3]
\end{Block}


\section{牛顿第二定律}
\begin{Point}
\lipsum[2]
\end{Point}

\begin{Case}
\item \lipsum[1][3]
\item \lipsum[1][3]
\item \lipsum[1][3]
\item \lipsum[1][3]
\end{Case}

\begin{Link}
\zhlipsum[1]
\end{Link}




\section{力学单位制}
\begin{Point}
\lipsum[2]
\end{Point}

\begin{Case}
\item \lipsum[1][3]
\item \lipsum[1][3]
\item \lipsum[1][3]
\item \lipsum[1][3]
\end{Case}

\begin{Link}
\zhlipsum[1]
\end{Link}




\section{失重和超重}

\begin{Point}
\lipsum[2]
\end{Point}

\begin{Case}
\item \lipsum[1][3]
\item \lipsum[1][3]
\item \lipsum[1][3]
\item \lipsum[1][3]
\end{Case}

\begin{Link}
\zhlipsum[1]
\end{Link}






\section{牛顿运动定律的应用}
\begin{Point}
\lipsum[2]
\end{Point}

\begin{Case}
\item \lipsum[1][3]
\item \lipsum[1][3]
\item \lipsum[1][3]
\item \lipsum[1][3]
\end{Case}

\begin{Link}
\zhlipsum[1]
\end{Link}





\section{牛顿运动定律的适用范围}
\begin{Point}
\lipsum[2]
\end{Point}

\begin{Case}
\item \lipsum[1][3]
\item \lipsum[1][3]
\item \lipsum[1][3]
\item \lipsum[1][3]
\end{Case}

\begin{Link}
\zhlipsum[1]
\end{Link}





\section{回顾与总结}

\begin{Point}
\lipsum[2]
\end{Point}

\begin{Case*}
\item \lipsum[1][3]
\item \lipsum[1][3]
\item \lipsum[1][3]
\item \lipsum[1][3]
\end{Case*}

\begin{Link}
\zhlipsum[1]
\end{Link}






\chaptersaying{\lipsum[1][8]}
\chapter{功和机械能}


\begin{Block}[章节引入]
\lipsum[1-3]
\end{Block}

\section{功和功率}

\begin{Point}
\lipsum[2]
\end{Point}

\begin{Case}
\item \lipsum[1][3]
\item \lipsum[1][3]
\item \lipsum[1][3]
\item \lipsum[1][3]
\end{Case}

\begin{Link}
\zhlipsum[1]
\end{Link}





\section{能量}

\begin{Point}
\lipsum[2]
\end{Point}

\begin{Case}
\item \lipsum[1][3]
\item \lipsum[1][3]
\item \lipsum[1][3]
\item \lipsum[1][3]
\end{Case}

\begin{Link}
\zhlipsum[1]
\end{Link}





\section{动能定理}


\begin{Point}
\lipsum[2]
\end{Point}

\begin{Case}
\item \lipsum[1][3]
\item \lipsum[1][3]
\item \lipsum[1][3]
\item \lipsum[1][3]
\end{Case}

\begin{Link}
\zhlipsum[1]
\end{Link}



\section{保守力的功}


\begin{Point}
\lipsum[2]
\end{Point}

\begin{Case}
\item \lipsum[1][3]
\item \lipsum[1][3]
\item \lipsum[1][3]
\item \lipsum[1][3]
\end{Case}

\begin{Link}
\zhlipsum[1]
\end{Link}



\section{机械能守恒定律}

\begin{Point}
\lipsum[2]
\end{Point}

\begin{Case}
\item \lipsum[1][3]
\item \lipsum[1][3]
\item \lipsum[1][3]
\item \lipsum[1][3]
\end{Case}

\begin{Link}
\zhlipsum[1]
\end{Link}





\section{功能关系}

\begin{Point}
\lipsum[2]
\end{Point}

\begin{Case}
\item \lipsum[1][3]
\item \lipsum[1][3]
\item \lipsum[1][3]
\item \lipsum[1][3]
\end{Case}

\begin{Link}
\zhlipsum[1]
\end{Link}




\section{回顾与总结}
\begin{Point}
\lipsum[2]
\end{Point}

\begin{Case*}
\item \lipsum[1][3]
\item \lipsum[1][3]
\item \lipsum[1][3]
\item \lipsum[1][3]
\end{Case*}

\begin{Link}
\zhlipsum[1]
\end{Link}





\chaptersaying{\lipsum[1][9]}
\chapter{动量}


\begin{Block}[章节引入]
\lipsum[1-3]
\end{Block}

\section{冲量和动量}
\begin{Point}
\lipsum[2]
\end{Point}

\begin{Case}
\item \lipsum[1][3]
\item \lipsum[1][3]
\item \lipsum[1][3]
\item \lipsum[1][3]
\end{Case}

\begin{Link}
\zhlipsum[1]
\end{Link}




\section{动量定理}

\begin{Point}
\lipsum[2]
\end{Point}

\begin{Case}
\item \lipsum[1][3]
\item \lipsum[1][3]
\item \lipsum[1][3]
\item \lipsum[1][3]
\end{Case}

\begin{Link}
\zhlipsum[1]
\end{Link}



\section{动量守恒定律}

\begin{Point}
\lipsum[2]
\end{Point}

\begin{Case}
\item \lipsum[1][3]
\item \lipsum[1][3]
\item \lipsum[1][3]
\item \lipsum[1][3]
\end{Case}

\begin{Link}
\zhlipsum[1]
\end{Link}




\section{碰撞和反冲}

\begin{Point}
\lipsum[2]
\end{Point}

\begin{Case}
\item \lipsum[1][3]
\item \lipsum[1][3]
\item \lipsum[1][3]
\item \lipsum[1][3]
\end{Case}

\begin{Link}
\zhlipsum[1]
\end{Link}



\section{回顾与总结}

\begin{Point}
\lipsum[2]
\end{Point}

\begin{Case*}
\item \lipsum[1][3]
\item \lipsum[1][3]
\item \lipsum[1][3]
\item \lipsum[1][3]
\end{Case*}

\begin{Link}
\zhlipsum[1]
\end{Link}




\part{振动和波}


\chaptersaying{\lipsum[1][9]}
\chapter{机械振动}


\begin{Block}[章节引入]
\lipsum[1-3]
\end{Block}

\section{简谐运动}
\begin{Point}
\lipsum[2]
\end{Point}

\begin{Case}
\item \lipsum[1][3]
\item \lipsum[1][3]
\item \lipsum[1][3]
\item \lipsum[1][3]
\end{Case}

\begin{Link}
\zhlipsum[1]
\end{Link}

\begin{Definition}[简谐运动]
物体受力大小与位移成正比,而方向相反,人们把具有这种特征的运动称为简谐运动。
\begin{equation*}
\bm F=-k\bm x
\end{equation*}\par
其中,$\bm F$~称为回复力。
\end{Definition}

\begin{Definition*}[简谐运动]
物体受力大小与位移成正比,而方向相反,人们把具有这种特征的运动称为简谐运动。
\begin{equation*}
\bm F=-k\bm x
\end{equation*}\par
其中,$\bm F$~称为回复力。
\end{Definition*}



\section{简谐运动的描述}
\begin{Point}
\lipsum[2]
\end{Point}

\begin{Case}
\item \lipsum[1][3]
\item \lipsum[1][3]
\item \lipsum[1][3]
\item \lipsum[1][3]
\end{Case}

\begin{Link}
\zhlipsum[1]
\end{Link}




\section{简谐运动的回复力与能量}
\begin{Point}
\lipsum[2]
\end{Point}

\begin{Case}
\item \lipsum[1][3]
\item \lipsum[1][3]
\item \lipsum[1][3]
\item \lipsum[1][3]
\end{Case}

\begin{Link}
\zhlipsum[1]
\end{Link}




\section{简谐运动实例}
\begin{Point}
\lipsum[2]
\end{Point}

\begin{Case}
\item \lipsum[1][3]
\item \lipsum[1][3]
\item \lipsum[1][3]
\item \lipsum[1][3]
\end{Case}

\begin{Link}
\zhlipsum[1]
\end{Link}


\section{受迫振动和共振}
\begin{Point}
\lipsum[2]
\end{Point}

\begin{Case}
\item \lipsum[1][3]
\item \lipsum[1][3]
\item \lipsum[1][3]
\item \lipsum[1][3]
\end{Case}

\begin{Link}
\zhlipsum[1]
\end{Link}

\section{回顾与总结}

\begin{Point}
\lipsum[2]
\end{Point}

\begin{Case*}
\item \lipsum[1][3]
\item \lipsum[1][3]
\item \lipsum[1][3]
\item \lipsum[1][3]
\end{Case*}

\begin{Link}
\zhlipsum[1]
\end{Link}





\chaptersaying{\lipsum[1][9]}
\chapter{机械波}


\begin{Block}[章节引入]
\lipsum[1-3]
\end{Block}

\section{波的形成}
\begin{Point}
\lipsum[2]
\end{Point}

\begin{Case}
\item \lipsum[1][3]
\item \lipsum[1][3]
\item \lipsum[1][3]
\item \lipsum[1][3]
\end{Case}

\begin{Link}
\zhlipsum[1]
\end{Link}



\section{波的描述}
\begin{Point}
\lipsum[2]
\end{Point}

\begin{Case}
\item \lipsum[1][3]
\item \lipsum[1][3]
\item \lipsum[1][3]
\item \lipsum[1][3]
\end{Case}

\begin{Link}
\zhlipsum[1]
\end{Link}




\section{波的反射、折射与衍射}
\begin{Point}
\lipsum[2]
\end{Point}

\begin{Case}
\item \lipsum[1][3]
\item \lipsum[1][3]
\item \lipsum[1][3]
\item \lipsum[1][3]
\end{Case}

\begin{Link}
\zhlipsum[1]
\end{Link}




\section{波的干涉}
\begin{Point}
\lipsum[2]
\end{Point}

\begin{Case}
\item \lipsum[1][3]
\item \lipsum[1][3]
\item \lipsum[1][3]
\item \lipsum[1][3]
\end{Case}

\begin{Link}
\zhlipsum[1]
\end{Link}


\section{多普勒效应}
\begin{Point}
\lipsum[2]
\end{Point}

\begin{Case}
\item \lipsum[1][3]
\item \lipsum[1][3]
\item \lipsum[1][3]
\item \lipsum[1][3]
\end{Case}

\begin{Link}
\zhlipsum[1]
\end{Link}



\section{回顾与总结}

\begin{Point}
\lipsum[2]
\end{Point}

\begin{Case*}
\item \lipsum[1][3]
\item \lipsum[1][3]
\item \lipsum[1][3]
\item \lipsum[1][3]
\end{Case*}

\begin{Link}
\zhlipsum[1]
\end{Link}




\part*{自我检测}

\begin{Test}
\chapter{第~3-4~章检测题}

\begin{multicols}{2}

\vspace*{-1.0cm}

\subsection{单项选择题}

\begin{QsNum}
\item \lipsum[1][1]
\item \lipsum[1][1]
\item \lipsum[1][1]
\item \lipsum[1][1]
\item \lipsum[1][1]
\item \lipsum[1][1]
\item \lipsum[1][1]
\item \lipsum[1][1]
\item \lipsum[1][1]
\item \lipsum[1][1]
\item \lipsum[1][1]
\item \lipsum[1][1]
\item \lipsum[1][1]
\item \lipsum[1][1]
\item \lipsum[1][1]
\item \lipsum[1][1]
\item \lipsum[1][1]
\item \lipsum[1][1]
\item \lipsum[1][1]
\item \lipsum[1][1]
\end{QsNum}

\subsection{多项选择题}

\begin{QsNum}
\item \lipsum[1][1]
\item \lipsum[1][1]
\item \lipsum[1][1]
\item \lipsum[1][1]
\item \lipsum[1][1]
\item \lipsum[1][1]
\item \lipsum[1][1]
\item \lipsum[1][1]
\item \lipsum[1][1]
\item \lipsum[1][1]
\end{QsNum}

\subsection{实验题}

\zhlipsum[3]

\subsection{计算题}

\zhlipsum[4-6]

\end{multicols}


\chapter{第~5-6~章检测题}



\chapter{第~7-9~章检测题}


\chapter{第~10-11~章检测题}


\end{Test}



\tcbstoprecording




\begin{Appendix}

\part*{附录}


\chapter{答案与提示}
\tcbinputrecords

\printbib*{reference}

\end{Appendix}




\printindex[noun]

\printindex[video]






\lipsum


\backmatter


\chaptersubtitle{补充说明}
\chapter*{后记}
\lipsum
\lipsum



\begin{box6}{自选盒子}
\lipsum[2]
\end{box6}










\makeback[][barcode.pdf]






\makespine[1.0cm]



\end{document}



