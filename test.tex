% !TEX program = bibtex
% !Mode:: "TeX:UTF-8"
\documentclass[color=COLORFUL]{textbook-cn}%openany
\usepackage{zhlipsum,lipsum}%乱数假文(测试文字)
\graphicspath{{./figure/}{./figures/}{./image/}{./images/}{./graphics/}{./graphic/}{./pictures/}{./picture/}}%提供多种图片路径

%%定义中文索引命令,#2是中文,#3是全拼(首字母)%%
\NewDocumentCommand{\zhindex}{omm}{\IfValueTF{#1}{\index[#1]{#3@#2}}{\index{#3@#2}}}
%%预置索引栏目%%
\makeindex[name=noun,title=名词索引,columns=3,columnsep=2.0em,intoc,options=-s indexstyle.ist]
\makeindex[name=video,title=视频索引,columns=2,columnsep=2.0em,options=-s indexstyle.ist]
%%设置英文索引命令%%
\newcommand{\indexnoun}[1]{\index[noun]{#1}}
\newcommand{\indexvideo}[1]{\index[video]{#1}}
%%设置中文索引命令%%
\newcommand{\zhindexnoun}[2]{\zhindex[noun]{#1}{#2}}
\newcommand{\zhindexvideo}[2]{\zhindex[video]{#1}{#2}}

%%双栏题目环境%%
\newenvironment{McQsNum}{\begin{multicols}{2}\begin{enumerate}[label=\protect\cir{\arabic*},itemindent=0em]}
{\end{enumerate}\end{multicols}\colorlet{ListColor}{ColorA}}


%\externaldocument{Ch1}%跨文件交叉引用
\series{物理学系列教材}
\version{三}
\title{量子力学}
\englishtitle{Quantum Mechanics}
\subtitle{夸克和中微子}
\englishsubtitle{Quark and Neutrino}
\author{作者}
\pressname{出版社}
\presslogo{gj.pdf}
\coverimage{coverimage.pdf}
\backimage{coverimage.pdf}


%\setmainfont{Times New Roman}

%%%%%%%%%%%%%%%%%%%%%%%%%%%%%%%%%%%%%%%%%%%%%%%%%%%%%%%%%%%%%%%%%%%%%%%%%%%%%%%%%%%
\begin{document}

\makecover


\maketitle


\frontmatter



\chapter*{序言}
\subsubsection{关于作者}\footnote{英明睿智的作者}
\lipsum
\subsubsection{关于本书}


\begin{center}
\scalebox{8}{\TextBook}\\[7.5pt]
\scalebox{8}{\TextBook*}
\end{center}

\chapter*{写在前面}
\lipsum


\chapter{写在前面}
\lipsum

\nomenclature{$\bm a$}{加速度}
\nomenclature{$\bm a^*$}{平均加速度}
\nomenclature{$\bm a_a$}{绝对加速度}
\nomenclature{$\bm a_C$}{质心加速度}
\nomenclature{$\bm a_e$}{牵连加速度}
\nomenclature{$\bm a_n$}{法向加速度}
\nomenclature{$\bm a_t$}{切向加速度}
\nomenclature{$A$}{面积,自由振动振幅}
\nomenclature{$\bm b$}{副法线基矢量}
\nomenclature{$\rm d$}{微分符号}
\nomenclature{$d$}{直径}
\nomenclature{$C$}{质心}
\nomenclature{$\bm e$}{单位矢量}
\nomenclature{$E$}{总机械能}
\nomenclature{$f$}{动摩擦因数,频率}
\nomenclature{$f_s$}{静摩擦因数}
\nomenclature{$\bm F$}{力}
\nomenclature{$\bm F_I$}{达朗贝尔惯性力}
\nomenclature{$\bm F_{IC}$}{科氏惯性力}
\nomenclature{$\bm F_{Ie}$}{牵连惯性力}
\nomenclature{$\bm F_N$}{法向约束力}
\nomenclature{$\bm F_P$}{主动力}
\nomenclature{$\bm F_R$}{力系的主矢,全反力}
\nomenclature{$\bm F_S$}{桁架中杆的内力}
\nomenclature{$\bm F_T$}{绳的张力}
\nomenclature{$\bm g$}{重力加速度}
\nomenclature{$h$}{高度}
\nomenclature{$\bm i$}{$x$~轴的基矢量}
\nomenclature{$\bm I$}{冲量}
\nomenclature{$\bm j$}{$y$~轴的基矢量}
\nomenclature{$\bm J_z$}{刚体对~$z$~轴的转动惯量}
\nomenclature{$k$}{弹簧刚度系数}
\nomenclature{$\bm k$}{$z$~轴的基矢量}
\nomenclature{$l$}{长度}
\nomenclature{$\bm L_C$}{刚体对质心的动量矩}
\nomenclature{$\bm L_O$}{刚体对点~$O$~的动量矩}
\nomenclature{$m$}{质量}
\nomenclature{$\bm M$}{力偶矩}
\nomenclature{$\bm M_I$}{惯性力系的主矩}
\nomenclature{$\bm M_O(F)$}{$F$~对点~$O$~轴的矩}
\nomenclature{$M_z(\bm F)$}{$F$~对~$z$~轴的矩}
\nomenclature{$n$}{质点数目}
\nomenclature{$\bm n$}{法线基矢量}
\nomenclature{$\bm p$}{动量}
\nomenclature{$q$}{载荷集度}
\nomenclature{$r$}{半径}
\nomenclature{$\bm r$}{矢径}
\nomenclature{$\bm r_C$}{质心的矢径}
\nomenclature{$\bm r_M$}{点~$M$~的矢径}
\nomenclature{$R$}{半径}
\nomenclature{$s$}{弧坐标}
\nomenclature{$t$}{时间}
\nomenclature{$T$}{动能,周期}
\nomenclature{$T_r$}{相对运动动能}
\nomenclature{$U$}{势力函数}
\nomenclature{$\bm v$}{速度}
\nomenclature{$\bm v^*$}{平均速度}
\nomenclature{$\bm v_a$}{绝对速度}
\nomenclature{$\bm v_C$}{质心速度}
\nomenclature{$\bm v_e$}{牵连速度}
\nomenclature{$\bm v_r$}{相对速度}
\nomenclature{$V$}{势能,体积}
\nomenclature{$W$}{力的功}
\nomenclature{$\bm W$}{重力}
\nomenclature{$x,y,z$}{直角坐标}
\nomenclature{$\bm\alpha$}{角加速度}
\nomenclature{$\alpha,\beta,\gamma$}{方位角}
\nomenclature{$\theta,\phi$}{常用角}
\nomenclature{$\delta$}{滚阻系数}
\nomenclature{$\updelta$}{变分符号}
\nomenclature{$\Delta$}{增量符号}
\nomenclature{$\kappa$}{曲率}
\nomenclature{$\kappa^*$}{平均曲率}
\nomenclature{$\lambda$}{弹簧的变形}
\nomenclature{$\rho$}{密度,曲率半径}
\nomenclature{$\bm\rho$}{相对矢径}
\nomenclature{$\bm\rho_C$}{质心的相对矢径}
\nomenclature{$\bm\tau$}{切线基矢量}
\nomenclature{$\phi_m$}{摩擦角}
\nomenclature{$\omega_0$}{固有角频率}
\nomenclature{$\bm\omega$}{角速度}


\chaptersubtitle{符号表}
\printnomenclature


\tableofcontents*

\listoffigures*


\listoftables*

\mainmatter
\tcbstartrecording






\chapter{前置章节}
\lipsum\lipsum\lipsum


\part{运动学}



\chapter{绪论\quad 测试文字}


\lipsum
\begin{figure}[htbp]
\centering\includegraphics[width=4\linewidth/5]{example-image-a}
\caption{极光}
\end{figure}




\chapter{第一主族\quad 碱金属}


\subsubsection*{拉格朗日乘子法拉格朗日乘子法拉格朗日乘子法拉格朗日乘子法拉格朗日乘子法拉格朗日乘子法}
\lipsum[1-3]

\subsubsection*{欧拉公式}
\lipsum[1-2]

\subsubsection*{费马大定理}
\lipsum[2-3]



\section{钠及其化合物}

\begin{Point}
\lipsum[2]
\end{Point}

\begin{Case}
\item 碳酸氢铵受热分解实验
\item 氮气的性质和用途
\item 铁的锈蚀实验
\end{Case}


\begin{Paracol}
\subsection{Look and Read}
\lipsum[3]

\subsubsection{化学帮我们正确认识物质}

\lipsum[1-3]
\DeepThink*{\lipsum[1][1]}
\RelaInfo*{\lipsum[1][1]}
\Remark*{\lipsum[1][1]}

\begin{Definition}[定义名称]
\lipsum[1][1-5]
\end{Definition}

\lipsum[2]



\begin{Axiom}[公理名称公理名称公理名称公理名称公理名称]
\lipsum[1][1-4]
\end{Axiom}

\begin{Axiom*}[公理名称公理名称公理名称公理名称公理名称]
\lipsum[1][1-4]
\end{Axiom*}

Just like Definition~\ref{Def.3.1.1} on page \pageref{Def.3.1.1}

\lipsum[2]
\DeepThink*{\lipsum[1][1]}
\RelaInfo*{\lipsum[1][1]}
\Remark*{\lipsum[1][1]}

\begin{Corollary}
\lipsum[2][1-7]
\end{Corollary}




\begin{Review}
\lipsum[2][1-4]
\end{Review}

\begin{Discussion}
\lipsum[1][1-4]
\end{Discussion}



\begin{Warning}
\lipsum[1][1-3]
\end{Warning}

\lipsum[2]\zhindexnoun{安培}{anpei}



\subsubsection{化学指导人类合理利用资源化学指导人类合理利用资源化学指导人类合理利用资源}



\begin{Theorem}[定理名称]
\lipsum[1][2-6]
\end{Theorem}


\begin{Proof}
\lipsum[1]
\end{Proof}




\begin{Lemma}[引理名称]
\zhlipsum[2]
\end{Lemma}


\begin{Proposition}[命题名称命题名称命题名称命题名称命题名称]
\zhlipsum[2]
\end{Proposition}

\begin{Warning}
\lipsum[1][1-3]
\end{Warning}


\subsubsection{化学促进科学技术的发展}

\lipsum[1-2]



\Example{
\lipsum[1][1-5]\\[5.0pt]
{\centering\includegraphics[width=\linewidth/2]{example-image-a}
\figcaption{例题图}}
}

\Answer{\lipsum[1][1-7]}




\begin{Definition*}[定义名称]
\lipsum[1][1-3]
\end{Definition*}

Just like \ref{Def.3.1.1} on page \pageref{Def.3.1.1}





\begin{Link}
\tcbsubtitle{环形电流假说}
\zhlipsum
\end{Link}



\lipsum[2]

\begin{Practice}
\lipsum[1][1-5]
\end{Practice}


\subsection{课时名}
\lipsum[3]

\subsubsection{标题名称}
\lipsum[1]

\Example{\lipsum[1][1-5]}
\Answer{\lipsum[1][1-5]}




\begin{Method}
\lipsum[1][1-8]
\end{Method}

\marginpic{qrcode.pdf}{图片名}


\Example{\zhlipsum[1]}
\Answer{\lipsum[1][1-5]}
\Answer*{\lipsum[1][1-5]}

\Variety{\lipsum[1][1-5]}
\Answer{\lipsum[1][1-5]}





\begin{Mind}[控制变量]
\lipsum[2]
\end{Mind}




\begin{Display}[向心力]
\lipsum[2]
\tcbsubtitle{注意事项}
\lipsum[2]\lipsum[2][1-2]
\end{Display}


\begin{Application}
\tcbsubtitle{案例名称案例名称案例名称}
\lipsum[2]
\tcbsubtitle{分析与解答}
\lipsum[2]
\end{Application}

\subsection{课时名课时名课时名课时名}
\paragraph{段落测试文字}
\lipsum[3]

\subsubsection{标题名称}
\lipsum[1]

\begin{History}
\lipsum[1-2]
\end{History}
%\DeepThink{请思考以下问题请思考以下问题请思考以下问题}


\begin{STS}
\lipsum[1-2]
\end{STS}



\begin{Information}
\lipsum[2]
\end{Information}


\subsubsection{标题名称}


\lipsum[1]

\end{Paracol}


\begin{Application}
\lipsum[2][1-3]
\end{Application}

\begin{Link}
\lipsum[2][1-3]
\end{Link}

\begin{Lemma}[万有引力定律]
\lipsum[2][1-3]
\end{Lemma}



\begin{Theorem}[动量定理]
\lipsum[2][1-3]
\end{Theorem}


\begin{Warning}
\lipsum[2][1-3]
\end{Warning}

\begin{Mind}[控制变量]
\lipsum[2][1-3]
\end{Mind}



\subsection{思考题}

\begin{Thinking}
\begin{QsNum}
\item \lipsum[1][1-2]
\item \lipsum[1][1-2]
\item \lipsum[1][1-2]
\item \lipsum[1][1-2]
\item \lipsum[1][1-2]
\item \lipsum[1][1-2]
\item \lipsum[1][1-2]
\item \lipsum[1][1-2]
\end{QsNum}
\tcblower
\lipsum[1]
\end{Thinking}






\begin{Exercise}
\tcbsubtitle{A~组}
\begin{QsNum}
\item \lipsum[1][1-2]
\item \lipsum[1][1-2]
\item \lipsum[1][1-2]
\item \lipsum[1][1-2]
\item \lipsum[1][1-2]
\item \lipsum[1][1-2]
\item \lipsum[1][1-2]
\item \lipsum[1][1-2]
\end{QsNum}
\tcbsubtitle{B~组}
\begin{QsNum}
\item \lipsum[1][1-2]
\item \lipsum[1][1-2]
\item \lipsum[1][1-2]
\item \lipsum[1][1-2]
\item \lipsum[1][1-2]
\end{QsNum}
\tcbsubtitle{C~组}
\begin{QsNum}
\item \lipsum[1][1-2]
\item \lipsum[1][1-2]
\item \lipsum[1][1-2]
\item \lipsum[1][1-2]
\item \lipsum[1][1-2]
\item \lipsum[1][1-2]
\item \lipsum[1][1-2]
\item \lipsum[1][1-2]
\end{QsNum}
\tcblower
\lipsum[1]
\end{Exercise}




\begin{CppBox}{有限元分析}
void FN( double A, double B, double *N, int *Ele );
void FNA( double A, double B, double *NA, int *Ele );
void FNB( double A, double B, double *NB, int *Ele );
double Inv_jaco( double A, double B, double *xo, double *yo, double *Nx, double *Ny, int *Ele );
void eld( double E, double u, double (*D)[3], int Iopt );
void CacuEKF( double E, double u, double Thick , double *xo, double *yo, double (*stress)[3], double *uo, double *vo, double *ef, double *ek, int *Ele, int nGauss , int Iopt );
void AssembleUF( double *ef, double (*UF)[2], int *Ele );
void skdd( int *jd, int (*Element)[8], int NN, int ND, int NE );
void SetStiff( int i, int j, double Value , double *Stiff , int *jd );
double GetStiff( int i, int j, double *Stiff , int *jd );
void AssembleK( double *ek, int *je, int EFN, double *Stiff, int *jd );
void Fix( int k, double a, double *Stiff , double *Load , int *jd, int NF );
void LLT( int *jd, double *zk, double *F, int NF );
void CacuSS( double E, double u, double *xo, double *yo, double *uo, double *vo, double (*SS)[4], int *Ele, int Iopt );
\end{CppBox}


\begin{PythonBox}{ABAQUS~自动批处理}
import numpy as np
import pandas as pd
inp_file = open('C:/Users/17939/Desktop/inp/Job.inp') # 
origin_data = inp_file.read() # 
inp_file.close() # 
# 生成新文件数
file_num = 15
# 
delta_x = 17.5
delta_y = 17.5
# 
parameter_name = ['cailiaomidu', 'tanxingmoliang', 'bosongbi', 
                  'chusudu', 'hengxiangpianyi', 'weizhibianliang', 
                  'zongxiangpianyi', 'gaodupianyi', 'zongshijian', 'shijiandanwei']
# 
parameter_value = [np.linspace(7850, 7850, file_num), # 
                   np.linspace(2.06e+11, 2.06e+11, file_num), # 
                   np.linspace(0.3, 0.3, file_num), # 
                   np.linspace(-500, -500, file_num), # 
                   np.linspace(-delta_x, delta_x, file_num), # 
                   np.linspace(-delta_x + 1, delta_x + 1, file_num), # 
                   np.linspace(-delta_y, delta_y, file_num), # 
                   np.linspace(0, 0, file_num), # 
                   np.linspace(5, 5, file_num), # 
                   np.linspace(1e-4, 1e-4, file_num)] # 
# 
for n in range(0, file_num): # 
    for m in range(0, file_num):
        temp_data = origin_data # 
        temp_data = temp_data.replace('hengxiangpianyi', str(parameter_value[4][n])) # 
        temp_data = temp_data.replace('weizhibianliang', str(parameter_value[5][n])) # 
        temp_data = temp_data.replace('zongxiangpianyi', str(parameter_value[6][m])) # 
        new_data = temp_data
        if 2.5 * n - delta_x < 0:
            x = '_' + str(abs(2.5 * n - delta_x))
        else:
            x = str(2.5 * n - delta_x)
        if 2.5 * m - delta_y < 0:
            y = '_' + str(abs(2.5 * m - delta_y))
        else:
            y = str(2.5 * m - delta_y)
        new_file = open('C:/Users/17939/Desktop/inp/TEST/'+ x + 'to' + y + '.inp', 'w') # 
        new_file.write(new_data) # 
        new_file.close() # 
\end{PythonBox}


\begin{Proposition*}[命题名称命题名称命题名称命题名称命题名称]
\zhlipsum[2]
\end{Proposition*}


\ExerciseTitle{基础巩固}
\lipsum

\ExerciseTitle[ColorB]{能力提升}
\lipsum



\begin{Topic}
\section{专题名专题名专题名专题名专题名专题名专题名专题名专题名}
\begin{Paracol}
\subsection{题型名称题型名称}

\subsubsection{方法解析}
\lipsum[2]


\subsubsection{典型例题}
\lipsum[2]

\Example{\lipsum[1][1-4]}
\Answer{\lipsum[1][1-4]}
\Answer*{\lipsum[1][1-4]}

\Example{\lipsum[1][1-4]}
\Answer{\lipsum[1][1-4]}
\Answer*{\lipsum[1][1-4]}

\subsection{题型名称}

\subsubsection{方法解析}
\lipsum[1-2]

\subsubsection{典型例题}
\lipsum[2]

\Example{\lipsum[1][1-4]}
\Answer{\lipsum[1][1-4]}
\Answer*{\lipsum[1][1-4]}

\Example{\lipsum[1][1-4]}
\Answer{\lipsum[1][1-4]}
\Answer*{\lipsum[1][1-4]}

\end{Paracol}


\begin{Specific}
\begin{QsNum}
\item \lipsum[1][1]
\xx{\lipsum[1][3]}{\lipsum[1][3]}{\lipsum[1][3]}{\lipsum[1][3]}
\item \lipsum[1][1]
\xx{\lipsum[1][3]}{\lipsum[1][3]}{\lipsum[1][3]}{\lipsum[1][3]}
\item \lipsum[1][1]
\xx{\lipsum[1][3]}{\lipsum[1][3]}{\lipsum[1][3]}{\lipsum[1][3]}
\item \lipsum[1][1]
\xx{polar}{saturn}{mars}{venus}
\item \lipsum[1][1]
\xx{\lipsum[1][3]}{\lipsum[1][3]}{\lipsum[1][3]}{\lipsum[1][3]}
\item \lipsum[1][1]
\xx{\lipsum[1][3]}{\lipsum[1][3]}{\lipsum[1][3]}{\lipsum[1][3]}
\item \lipsum[1][1]
\xx{\lipsum[1][3]}{\lipsum[1][3]}{\lipsum[1][3]}{\lipsum[1][3]}
\item \lipsum[1][1]
\xx{\lipsum[1][3]}{\lipsum[1][3]}{\lipsum[1][3]}{\lipsum[1][3]}
\item \lipsum[1][1]
\xx{polar}{saturn}{mars}{venus}
\item \lipsum[1][1]
\xx{\lipsum[1][3]}{\lipsum[1][3]}{\lipsum[1][3]}{\lipsum[1][3]}
\end{QsNum}
\tcblower
\lipsum[1]
\end{Specific}



\end{Topic}

\begin{Topic}
\section{专题名}
\begin{Paracol}
\subsection{题型名称}

\subsubsection{方法解析}
\lipsum[1]

\subsubsection{典型例题}

\Example{\lipsum[1][1-4]}
\Answer{\lipsum[1][1-4]}
\Answer*{\lipsum[1][1-4]}

\Example{\lipsum[1][1-4]}
\Answer{\lipsum[1][1-4]}
\Answer*{\lipsum[1][1-4]}

\end{Paracol}

\begin{Specific}
\begin{QsNum}
\item \lipsum[1][1]
\xx{\lipsum[1][3]}{\lipsum[1][3]}{\lipsum[1][3]}{\lipsum[1][3]}
\item \lipsum[1][1]
\xx{\lipsum[1][3]}{\lipsum[1][3]}{\lipsum[1][3]}{\lipsum[1][3]}
\item \lipsum[1][1]
\xx{\lipsum[1][3]}{\lipsum[1][3]}{\lipsum[1][3]}{\lipsum[1][3]}
\item \lipsum[1][1]
\xx{polar}{saturn}{mars}{venus}
\item \lipsum[1][1]
\xx{\lipsum[1][3]}{\lipsum[1][3]}{\lipsum[1][3]}{\lipsum[1][3]}
\item \lipsum[1][1]
\xx{\lipsum[1][3]}{\lipsum[1][3]}{\lipsum[1][3]}{\lipsum[1][3]}
\item \lipsum[1][1]
\xx{\lipsum[1][3]}{\lipsum[1][3]}{\lipsum[1][3]}{\lipsum[1][3]}
\item \lipsum[1][1]
\xx{\lipsum[1][3]}{\lipsum[1][3]}{\lipsum[1][3]}{\lipsum[1][3]}
\item \lipsum[1][1]
\xx{polar}{saturn}{mars}{venus}
\item \lipsum[1][1]
\xx{\lipsum[1][3]}{\lipsum[1][3]}{\lipsum[1][3]}{\lipsum[1][3]}
\end{QsNum}
\tcblower
\lipsum[1]
\end{Specific}

\end{Topic}


\section{实验:实验名}
\begin{Point}
\lipsum[2]
\end{Point}

\begin{Case}
\item 碳酸氢铵受热分解实验
\item 氮气的性质和用途
\item 铁的锈蚀实验
\end{Case}


\begin{Corollary*}[推论名称]
\lipsum[2][1-7]
\end{Corollary*}



\chapter*{测试章节}
\lipsum\lipsum




\section{锂及其化合物}

\begin{Point}
\lipsum[2]
\end{Point}

\begin{Case}
\item 碳酸氢铵受热分解实验
\item 氮气的性质和用途
\item 铁的锈蚀实验
\end{Case}

\begin{Paracol}
\subsection{锂单质的性质}
\lipsum[2][1-8]
\subsubsection{锂的物理性质}
\lipsum[1-2]
\subsubsection{锂的化学性质}
\lipsum[1-2]
\subsection{锂的化合物}
\lipsum[2][1-8]
\subsubsection{氢氧化锂}
\lipsum[1-2]
\subsubsection{氧化锂}
\lipsum[1-3]

\begin{Block}[文字标题]
\lipsum[2]
\end{Block}


\begin{Block}
\lipsum[2]
\end{Block}

\subsection{锂电池}
\lipsum[1-2]
\subsection{锂的其他应用}
\lipsum[1-2]
\begin{Definition}[定义名称]
\lipsum[1][1-3]
\end{Definition}

Just like \ref{Def.3.1.1} on page \pageref{Def.3.1.1}


\begin{Theorem*}[定理名称]
\lipsum[1][1-3]
\end{Theorem*}

Just like \ref{Thm.3.1.1} on page \pageref{Thm.3.1.1}

\end{Paracol}


\begin{Exercise}
\begin{QsNum}
\item \lipsum[1][1-2]
\item \lipsum[1][1-2]
\item \lipsum[1][1-2]
\item \lipsum[1][1-2]
\item \lipsum[1][1-2]
\item \lipsum[1][1-2]
\item \lipsum[1][1-2]
\item \lipsum[1][1-2]
\end{QsNum}
\tcblower
\lipsum[1]
\end{Exercise}



\section{钠化合物的性质}
\begin{Point}
\lipsum[2]
\end{Point}

\begin{Case}
\item 碳酸氢铵受热分解实验
\item 氮气的性质和用途
\item 铁的锈蚀实验
\end{Case}


\subsection{锂单质的性质}
\subsubsection{锂的物理性质锂的物理性质}
\lipsum[1-2]

\begin{Lemma*}[万有引力定律]
\lipsum[2][1-3]
\end{Lemma*}
\subsubsection{锂的化学性质}
\lipsum[1-2]
\subsection{锂的化合物}
\lipsum[3]
\subsubsection{氢氧化锂}
\lipsum[1-2]
\subsubsection{氧化锂}
\lipsum[1-2]
\subsection{锂电池}
\lipsum[1-2]
\subsection{锂的其他应用}
\lipsum[1-2]

\begin{Exercise}
\begin{QsNum}
\item \lipsum[1][1-2]
\item \lipsum[1][1-2]
\item \lipsum[1][1-2]
\item \lipsum[1][1-2]
\item \lipsum[1][1-2]
\item \lipsum[1][1-2]
\item \lipsum[1][1-2]
\item \lipsum[1][1-2]
\end{QsNum}
\tcblower
\lipsum[1]
\end{Exercise}


\section{回顾与总结}

\begin{Point*}
\lipsum[2]
\end{Point*}

\begin{Case*}
\item \lipsum[1][1]
\item \lipsum[1][1]
\item \lipsum[1][1]
\end{Case*}

\subsection{测试文字测试文字测试文字}


\subsubsection*{测试文字测试文字}
\lipsum[1-2]

\Example{\lipsum[1][1-5]}
\Answer{\lipsum[1][1-4]}
\Answer*{\lipsum[1][1-4]}

\Variety{\lipsum[1][1-5]}
\Answer{\lipsum[1][1-4]}
\Answer*{\lipsum[1][1-4]}

\subsubsection*{测试文字测试文字}
\lipsum[1-2]

\Example{\lipsum[1][1-5]}
\Answer{\lipsum[1][1-4]}
\Answer*{\lipsum[1][1-4]}


\Variety{\lipsum[1][1-5]}
\Answer{\lipsum[1][1-4]}
\Answer*{\lipsum[1][1-4]}


\subsection{测试文字测试文字测试文字测试文字测试文字测试文字测试文字}
\subsubsection*{测试文字测试文字}
\lipsum[1-2]

\Example{\lipsum[1][1-5]}
\Answer{\lipsum[1][1-4]}
\Answer*{\lipsum[1][1-4]}

\Variety{\lipsum[1][1-5]}
\Answer{\lipsum[1][1-4]}
\Answer*{\lipsum[1][1-4]}

\begin{Improve}
\tcbsubtitle{A~组}
\begin{QsNum}
\item \lipsum[1][1-4]
\item \lipsum[1][1-4]
\item \lipsum[1][1-4]
\item \lipsum[1][1-4]
\item \lipsum[1][1-4]
\item \lipsum[1][1-4]
\item \lipsum[1][1-4]
\item \lipsum[1][1-4]
\item \lipsum[1][1-4]
\end{QsNum}
\tcbsubtitle{B~组}
\begin{QsNum}
\item \lipsum[1][1-4]
\item \lipsum[1][1-4]
\item \lipsum[1][1-4]
\item \lipsum[1][1-4]
\item \lipsum[1][1-4]
\item \lipsum[1][1-4]
\item \lipsum[1][1-4]
\item \lipsum[1][1-4]
\item \lipsum[1][1-4]
\end{QsNum}
\tcbsubtitle{C~组}
\begin{QsNum}
\item \lipsum[1][1-4]
\item \lipsum[1][1-4]
\item \lipsum[1][1-4]
\item \lipsum[1][1-4]
\item \lipsum[1][1-4]
\item \lipsum[1][1-4]
\item \lipsum[1][1-4]
\item \lipsum[1][1-4]
\item \lipsum[1][1-4]
\end{QsNum}
\tcblower
\lipsum[1]
\end{Improve}


\lipsum





\part{静力学}

% !TEX program = bibtex

\chapter{开启化学之门}

\begin{Block}[章节引言]
\lipsum[1-4]
\end{Block}




\section{化学给我们带来什么}%小节名
\begin{Point}
\lipsum[2]
\end{Point}

\begin{Case}
\item 碳酸氢铵受热分解实验
\item 氮气的性质和用途
\item 铁的锈蚀实验
\end{Case}

\lipsum\cite{51}\label{conclusion}


\section{化学给我们带来什么}%小节名
\begin{Point}
\lipsum[2]
\end{Point}



\lipsum\cite{52}\cite{53}







\begin{Topic}
\section{专题名}
\begin{Paracol}
\subsection{题型名称}

\subsubsection{方法解析}
\lipsum[1]

\subsubsection{典型例题}
\Example{\lipsum[1][1-4]}
\Answer{\lipsum[1][1-4]}
\Answer*{\lipsum[1][1-4]}

\Example{\lipsum[1][1-4]}
\Answer{\lipsum[1][1-4]}
\Answer*{\lipsum[1][1-4]}

\end{Paracol}

\begin{Specific}
\begin{QsNum}
\item \lipsum[1][1]
\choice{\lipsum[1][3]}{\lipsum[1][3]}{\lipsum[1][3]}{\lipsum[1][3]}
\item \lipsum[1][1]
\choice{\lipsum[1][3]}{\lipsum[1][3]}{\lipsum[1][3]}{\lipsum[1][3]}
\item \lipsum[1][1]
\choice{\lipsum[1][3]}{\lipsum[1][3]}{\lipsum[1][3]}{\lipsum[1][3]}
\end{QsNum}
\tcblower
\lipsum[1]
\end{Specific}

\end{Topic}



\begin{Topic}
\section{专题名}

\begin{Paracol}
\subsection{题型名称}

\subsubsection{方法解析}
\lipsum[1]

\subsubsection{典型例题}
\Example{\lipsum[1][1-4]}
\Answer{\lipsum[1][1-4]}
\Answer*{\lipsum[1][1-4]}

\Example{\lipsum[1][1-4]}
\Answer{\lipsum[1][1-4]}
\Answer*{\lipsum[1][1-4]}

\end{Paracol}

\begin{Specific}
\begin{QsNum}
\item \lipsum[1][1]
\choice{\lipsum[1][3]}{\lipsum[1][3]}{\lipsum[1][3]}{\lipsum[1][3]}
\item \lipsum[1][1]
\choice{\lipsum[1][3]}{\lipsum[1][3]}{\lipsum[1][3]}{\lipsum[1][3]}
\item \lipsum[1][1]
\choice{\lipsum[1][3]}{\lipsum[1][3]}{\lipsum[1][3]}{\lipsum[1][3]}
\end{QsNum}
\tcblower
\lipsum[1]
\end{Specific}

\end{Topic}




\section{回顾与总结}
\begin{Point*}
\lipsum[1]
\end{Point*}


\subsection{测试文字}
\subsubsection{考点名称}

\lipsum[2-3]\cite{54}

\subsubsection{考点名称}

\lipsum[2-3]

\subsubsection{考点名称}

\lipsum[2-3]

\subsubsection{考点名称}

\lipsum[2-3]

\subsubsection{考点名称}

\lipsum[2-3]


\subsection*{测试不计数小节}

%\printbib{Reference}

\begin{Improve}
\tcbsubtitle{A~组}
\begin{QsNum}
\item \lipsum[1][1-4]
\item \lipsum[1][1-4]
\item \lipsum[1][1-4]
\item \lipsum[1][1-4]
\end{QsNum}
\tcbsubtitle{B~组}
\begin{QsNum}
\item \lipsum[1][1-4]
\item \lipsum[1][1-4]
\item \lipsum[1][1-4]
\item \lipsum[1][1-4]
\end{QsNum}
\tcbsubtitle{C~组}
\begin{QsNum}
\item \lipsum[1][1-4]
\item \lipsum[1][1-4]
\item \lipsum[1][1-4]
\item \lipsum[1][1-4]
\item \lipsum[1][1-4]
\item \lipsum[1][1-4]
\item \lipsum[1][1-4]
\item \lipsum[1][1-4]
\end{QsNum}
\tcblower
\lipsum[1]
\end{Improve}




















\begin{Quiz}
\section{期中测试}
\subsection{题目名称}
\lipsum[1-5]
\subsection{题目名称}
\lipsum[1-5]
\subsection{题目名称}
\lipsum[1-5]
\end{Quiz}


\begin{Quiz}
\section{期中测试}
\subsection{题目名称}
\lipsum[1-5]
\subsection{题目名称}
\lipsum[1-5]
\subsection{题目名称}
\lipsum[1-5]
\end{Quiz}


\chapter{常见的官能团}

\section{烷烃及其性质}
\begin{Point}
\lipsum[1]
\end{Point}

\subsection{测试文字}

\lipsum


\begin{Project}
\section{实验名}
\begin{Point}
\lipsum[2]
\end{Point}

\begin{Case}
\item 碳酸氢铵受热分解实验
\item 氮气的性质和用途
\item 铁的锈蚀实验
\end{Case}

\subsection{测试文字}
\lipsum
\begin{Definition}[定理名称]
\lipsum[2]
\end{Definition}

\begin{Application}
\tcbsubtitle{案例名称}
\lipsum[1-2]
\tcbsubtitle{问题分析}
\lipsum[3-4]
\end{Application}
\end{Project}


\section{取代反应}
\begin{Point}
\lipsum[2]
\end{Point}

\begin{Case}
\item 碳酸氢铵受热分解实验
\item 氮气的性质和用途
\item 铁的锈蚀实验
\end{Case}

\lipsum
\begin{Definition*}[定理名称定理名称定理名称定理名称定理名称定理名称]
\lipsum[2]
\end{Definition*}

\begin{Definition*}[定理名称]
\lipsum[2]
\end{Definition*}

\begin{Lemma}[引理名称]
\lipsum[2]
\end{Lemma}



\begin{Project}
\section{实验名}
\begin{Point}
\lipsum[2]
\end{Point}

\begin{Case}
\item 碳酸氢铵受热分解实验
\item 氮气的性质和用途
\item 铁的锈蚀实验
\end{Case}

\subsection{测试文字}
\lipsum
\begin{Definition}[定理名称]
\lipsum[1]
\end{Definition}

\begin{Lemma}[引理名称]
\lipsum[1]
\end{Lemma}
\end{Project}



\section{回顾与总结}
\begin{Point*}
\lipsum[1]
\end{Point*}

\begin{Case*}
\item \lipsum[1][1]
\item \lipsum[1][1]
\item \lipsum[1][1]
\item \lipsum[1][1]
\item \lipsum[1][1]
\item \lipsum[1][1]
\item \lipsum[1][1]
\end{Case*}


\lipsum[1-6]
\begin{Improve}
\tcbsubtitle{A~组}
\begin{QsNum}
\item \lipsum[1][1-4]
\item \lipsum[1][1-4]
\item \lipsum[1][1-4]
\item \lipsum[1][1-4]
\item \lipsum[1][1-4]
\item \lipsum[1][1-4]
\item \lipsum[1][1-4]
\item \lipsum[1][1-4]
\end{QsNum}
\tcbsubtitle{B~组}
\begin{QsNum}
\item \lipsum[1][1-4]
\item \lipsum[1][1-4]
\item \lipsum[1][1-4]
\item \lipsum[1][1-4]
\item \lipsum[1][1-4]
\item \lipsum[1][1-4]
\item \lipsum[1][1-4]
\item \lipsum[1][1-4]
\end{QsNum}
\tcblower
\lipsum[1]
\end{Improve}





\begin{Quiz}
\section{主观题}
\end{Quiz}



\begin{Project}
\section{实验名}

\begin{Theorem*}[定理名称]
\lipsum[1][1-3]
\end{Theorem*}
\end{Project}


\part*{自我检测}
\part*{自我检测自我检测}


\begin{Quiz}
\chapter{检测题}


\section{主观题}
\end{Quiz}






\begin{Test}
\chapter{第~1-2~章检测题}
\lipsum\lipsum\cite{7}\cite{6}
\subsection{题目}
\end{Test}

\begin{Quiz}
\chapter{第~3-4~章检测题}
\section{主观题}
\end{Quiz}



\tcbstoprecording


\begin{Appendix}






\part*{附录}

\chapter{参考答案}
\tcbinputrecords

\chapter{测试文字}

\chapter*{测试文字}




\section{测试文字}

\lipsum[1-4]

\section{测试文字}
\subsection{小节名称}
\lipsum[1-4]

\subsection{小节名称}
\lipsum[1-2]

\Example{\lipsum[1][1-6]}
\Example{\lipsum[1][1-6]}

\Variety{\lipsum[1][1-6]}
\Variety{\lipsum[1][1-6]}

\lipsum[3-4]

\section*{测试文字}
\lipsum[1-8]
\RelaInfo{\lipsum[1-2]}

\DeepThink{\lipsum[2]}

\Remark{\lipsum[2]}

\clearpage
Aachen\indexvideo{Aachen} aardvark\indexvideo{aardvark} aargh\indexvideo{aargh} Aarhus\indexvideo{Aarhus} Aaron\indexvideo{Aaron} Aaronvitch\indexvideo{Aaronvitch} Ababa\indexvideo{Ababa} aback\indexvideo{aback} abacus\indexvideo{abacus} abaft\indexvideo{abaft} abalone\indexvideo{abalone} abandon\indexvideo{abandon} abandoner\indexvideo{abandoner} abase\indexvideo{abase} abaser\indexvideo{abaser} abash\indexvideo{abash} abashed\indexvideo{abashed} abate\indexvideo{abate} abated\indexvideo{abated} abater\indexvideo{abater} abattoir\indexvideo{abattoir} Abba\indexvideo{Abba} abbess\indexvideo{abbess} abbey\indexvideo{abbey} abbot\indexvideo{abbot} Abbott\indexvideo{Abbott} abbreviate\indexvideo{abbreviate} abbreviated\indexvideo{abbreviated} abbreviation\indexvideo{abbreviation} abbé\indexvideo{abbé} ABC\indexvideo{ABC} abdicate\indexvideo{abdicate} abdication\indexvideo{abdication} abdomen\indexvideo{abdomen} abdominal\indexvideo{abdominal} abduct\indexvideo{abduct} abductee\indexvideo{abductee} abduction\indexvideo{abduction} abductor\indexvideo{abductor} Abdul\indexvideo{Abdul} Abe\indexvideo{Abe} abeam\indexvideo{abeam} Abel\indexvideo{Abel} Abelard\indexvideo{Abelard} Abelson\indexvideo{Abelson} Aberconwy\indexvideo{Aberconwy} Aberdeen\indexvideo{Aberdeen} Aberdeenshire\indexvideo{Aberdeenshire} Abernathy\indexvideo{Abernathy} aberrant\indexvideo{aberrant} aberration\indexvideo{aberration} Aberystwyth\indexvideo{Aberystwyth} abet\indexvideo{abet} abettor\indexvideo{abettor} abeyance\indexvideo{abeyance} 
Aachen\indexnoun{Aachen} aardvark\indexnoun{aardvark} aargh\indexnoun{aargh} Aarhus\indexnoun{Aarhus} Aaron\indexnoun{Aaron} Aaronvitch\indexnoun{Aaronvitch} Ababa\indexnoun{Ababa} aback\indexnoun{aback} abacus\indexnoun{abacus} abaft\indexnoun{abaft} abalone\indexnoun{abalone} abandon\indexnoun{abandon} abandoner\indexnoun{abandoner} abase\indexnoun{abase} abaser\indexnoun{abaser} abash\indexnoun{abash} abashed\indexnoun{abashed} abate\indexnoun{abate} abated\indexnoun{abated} abater\indexnoun{abater} abattoir\indexnoun{abattoir} Abba\indexnoun{Abba} abbess\indexnoun{abbess} abbey\indexnoun{abbey} abbot\indexnoun{abbot} Abbott\indexnoun{Abbott} abbreviate\indexnoun{abbreviate} abbreviated\indexnoun{abbreviated} abbreviation\indexnoun{abbreviation} abbé\indexnoun{abbé} ABC\indexnoun{ABC} abdicate\indexnoun{abdicate} abdication\indexnoun{abdication} abdomen\indexnoun{abdomen} abdominal\indexnoun{abdominal} abduct\indexnoun{abduct} abductee\indexnoun{abductee} abduction\indexnoun{abduction} abductor\indexnoun{abductor} Abdul\indexnoun{Abdul} Abe\indexnoun{Abe} abeam\indexnoun{abeam} Abel\indexnoun{Abel} Abelard\indexnoun{Abelard} Abelson\indexnoun{Abelson} Aberconwy\indexnoun{Aberconwy} Aberdeen\indexnoun{Aberdeen} Aberdeenshire\indexnoun{Aberdeenshire} Abernathy\indexnoun{Abernathy} aberrant\indexnoun{aberrant} aberration\indexnoun{aberration} Aberystwyth\indexnoun{Aberystwyth} abet\indexnoun{abet} abettor\indexnoun{abettor} abeyance\indexnoun{abeyance} abeyant\indexnoun{abeyant} abhor\indexnoun{abhor} abhorrence\indexnoun{abhorrence} abhorrent\indexnoun{abhorrent} abhorrer\indexnoun{abhorrer} abidance\indexnoun{abidance} abide\indexnoun{abide} abider\indexnoun{abider} Abidjan\indexnoun{Abidjan} Abigail\indexnoun{Abigail} Abilene\indexnoun{Abilene} ability\indexnoun{ability} abiogenesis\indexnoun{abiogenesis} abiogenic\indexnoun{abiogenic} abiotic\indexnoun{abiotic} abject\indexnoun{abject} abjection\indexnoun{abjection} abjectness\indexnoun{abjectness} abjuration\indexnoun{abjuration} abjure\indexnoun{abjure} ablate\indexnoun{ablate} ablation\indexnoun{ablation} ablaze\indexnoun{ablaze} able-bodied\indexnoun{able-bodied} able\indexnoun{able} abler\indexnoun{abler} abloom\indexnoun{abloom} ablution\indexnoun{ablution} abnegate\indexnoun{abnegate} abnegation\indexnoun{abnegation} abnormal\indexnoun{abnormal} abnormality\indexnoun{abnormality} aboard\indexnoun{aboard} abode\indexnoun{abode} abolish\indexnoun{abolish} abolition\indexnoun{abolition} abolitionism\indexnoun{abolitionism} abolitionist\indexnoun{abolitionist} abominable\indexnoun{abominable} abominate\indexnoun{abominate} abomination\indexnoun{abomination} aboriginal\indexnoun{aboriginal} Aboriginals\indexnoun{Aboriginals} aborigine\indexnoun{aborigine} abort\indexnoun{abort} aborter\indexnoun{aborter} abortion\indexnoun{abortion} abortionist\indexnoun{abortionist} abortive\indexnoun{abortive} abound\indexnoun{abound} about\indexnoun{about} above\indexnoun{above} aboveboard\indexnoun{aboveboard} aboveground\indexnoun{aboveground} abracadabra\indexnoun{abracadabra} abrade\indexnoun{abrade} abrader\indexnoun{abrader} Abraham\indexnoun{Abraham} Abram\indexnoun{Abram} abrasion\indexnoun{abrasion} abrasive\indexnoun{abrasive} abrasiveness\indexnoun{abrasiveness} abreaction\indexnoun{abreaction} abreast\indexnoun{abreast} abridge\indexnoun{abridge} abridged\indexnoun{abridged} abridger\indexnoun{abridger} abroad\indexnoun{abroad} abrogate\indexnoun{abrogate} abrogation\indexnoun{abrogation} abrogator\indexnoun{abrogator} abrupt\indexnoun{abrupt} abruptness\indexnoun{abruptness} abs\indexnoun{abs} abscess\indexnoun{abscess} abscissa\indexnoun{abscissa} abscissae\indexnoun{abscissae} abscission\indexnoun{abscission} abscond\indexnoun{abscond} abseil\indexnoun{abseil} abseiler\indexnoun{abseiler} absence\indexnoun{absence} absent-minded\indexnoun{absent-minded} absent-mindedness\indexnoun{absent-mindedness} absent\indexnoun{absent} absentee\indexnoun{absentee} absenteeism\indexnoun{absenteeism} absenter\indexnoun{absenter} absentia\indexnoun{absentia} absinthe\indexnoun{absinthe} absolute\indexnoun{absolute} absoluteness\indexnoun{absoluteness} absolution\indexnoun{absolution} absolutism\indexnoun{absolutism} absolve\indexnoun{absolve} absolver\indexnoun{absolver} absorb\indexnoun{absorb} absorbed\indexnoun{absorbed} absorbency\indexnoun{absorbency} absorbent\indexnoun{absorbent} absorbs\indexnoun{absorbs} absorption\indexnoun{absorption} absorptivity\indexnoun{absorptivity} abstain\indexnoun{abstain} abstemious\indexnoun{abstemious} abstemiousness\indexnoun{abstemiousness} abstention\indexnoun{abstention} abstinence\indexnoun{abstinence} abstinent\indexnoun{abstinent} abstract\indexnoun{abstract} abstracted\indexnoun{abstracted} abstractedness\indexnoun{abstractedness} abstracter\indexnoun{abstracter} abstraction\indexnoun{abstraction} abstractionism\indexnoun{abstractionism} abstractionist\indexnoun{abstractionist} abstractness\indexnoun{abstractness} abstractor\indexnoun{abstractor} abstruse\indexnoun{abstruse} abstruseness\indexnoun{abstruseness} absurd\indexnoun{absurd} absurdity\indexnoun{absurdity} absurdness\indexnoun{absurdness} Abu\indexnoun{Abu} Abuja\indexnoun{Abuja} abundance\indexnoun{abundance} abundant\indexnoun{abundant} abusable\indexnoun{abusable} abuse\indexnoun{abuse} abuser\indexnoun{abuser} abusive\indexnoun{abusive} abusiveness\indexnoun{abusiveness} abut\indexnoun{abut} abuzz\indexnoun{abuzz} abysmal\indexnoun{abysmal} abyss\indexnoun{abyss} abyssal\indexnoun{abyssal} Abyssinia\indexnoun{Abyssinia} Abyssinian\indexnoun{Abyssinian} Ac\indexnoun{Ac} AC\indexnoun{AC} acacia\indexnoun{acacia} academe\indexnoun{academe} academia\indexnoun{academia} academic\indexnoun{academic} academician\indexnoun{academician} academicianship\indexnoun{academicianship} academy\indexnoun{academy} Acadia\indexnoun{Acadia} acanthus\indexnoun{acanthus} Acapulco\indexnoun{Acapulco} ACAS\indexnoun{ACAS} acc.\indexnoun{acc.} accede\indexnoun{accede} accelerate\indexnoun{accelerate} acceleration\indexnoun{acceleration} accelerator\indexnoun{accelerator} accelerometer\indexnoun{accelerometer} accent\indexnoun{accent} accented\indexnoun{accented} accentual\indexnoun{accentual} accentuate\indexnoun{accentuate} accentuation\indexnoun{accentuation} accept\indexnoun{accept} acceptability\indexnoun{acceptability} acceptable\indexnoun{acceptable} acceptableness\indexnoun{acceptableness} acceptably\indexnoun{acceptably} acceptance\indexnoun{acceptance} acceptant\indexnoun{acceptant} accepted\indexnoun{accepted} acceptingness\indexnoun{acceptingness} acceptor\indexnoun{acceptor} access\indexnoun{access} accessibility\indexnoun{accessibility} accessible\indexnoun{accessible} accessibly\indexnoun{accessibly} accession\indexnoun{accession} accessors\indexnoun{accessors} accessory\indexnoun{accessory} accidence\indexnoun{accidence} accident-prone\indexnoun{accident-prone} accident\indexnoun{accident} accidental\indexnoun{accidental} acclaim\indexnoun{acclaim} acclaimer\indexnoun{acclaimer} acclamation\indexnoun{acclamation} acclimate\indexnoun{acclimate} acclimation\indexnoun{acclimation} acclimatisation\indexnoun{acclimatisation} acclimatise\indexnoun{acclimatise} acclimatization\indexnoun{acclimatization} acclimatize\indexnoun{acclimatize} acclivity\indexnoun{acclivity} accolade\indexnoun{accolade} accommodate\indexnoun{accommodate} accommodating\indexnoun{accommodating} accommodation\indexnoun{accommodation} accommodative\indexnoun{accommodative} accompanied\indexnoun{accompanied} accompanier\indexnoun{accompanier} accompany\indexnoun{accompany} accomplice\indexnoun{accomplice} accomplish\indexnoun{accomplish} accomplished\indexnoun{accomplished} accord\indexnoun{accord} accordance\indexnoun{accordance} accordant\indexnoun{accordant} accordion\indexnoun{accordion} accordionist\indexnoun{accordionist} accost\indexnoun{accost} account\indexnoun{account} accountability's\indexnoun{accountability's} accountability\indexnoun{accountability} accountable\indexnoun{accountable} accountably\indexnoun{accountably} accountancy\indexnoun{accountancy} accountant\indexnoun{accountant} accounted\indexnoun{accounted} accounting\indexnoun{accounting} accoutre\indexnoun{accoutre} Accra\indexnoun{Accra} accredit\indexnoun{accredit} accreditation\indexnoun{accreditation} accredited\indexnoun{accredited} accreted\indexnoun{accreted} accretion\indexnoun{accretion} accretive\indexnoun{accretive} accrual\indexnoun{accrual} accrue\indexnoun{accrue} acct\indexnoun{acct} acculturate\indexnoun{acculturate} acculturation\indexnoun{acculturation} accumulate\indexnoun{accumulate} accumulation\indexnoun{accumulation} accumulative\indexnoun{accumulative} accumulator\indexnoun{accumulator} accuracy\indexnoun{accuracy} accurate\indexnoun{accurate} accurately\indexnoun{accurately} accurateness\indexnoun{accurateness} accursed\indexnoun{accursed} accursedness\indexnoun{accursedness} accusal\indexnoun{accusal} accusation\indexnoun{accusation} accusative\indexnoun{accusative} accusatory\indexnoun{accusatory} accuse\indexnoun{accuse} accused\indexnoun{accused} accustom\indexnoun{accustom} accustomed\indexnoun{accustomed} accustomedness\indexnoun{accustomedness} ace\indexnoun{ace} acellular\indexnoun{acellular} acer\indexnoun{acer} acerbate\indexnoun{acerbate} acerbic\indexnoun{acerbic} acerbity\indexnoun{acerbity} acetaminophen\indexnoun{acetaminophen} acetate\indexnoun{acetate} acetic\indexnoun{acetic} acetogenic\indexnoun{acetogenic} acetone\indexnoun{acetone} acetylcholine\indexnoun{acetylcholine} acetylene\indexnoun{acetylene} Achaean\indexnoun{Achaean} ache\indexnoun{ache} ached\indexnoun{ached} achene\indexnoun{achene} aches\indexnoun{aches} achievable\indexnoun{achievable} achieve\indexnoun{achieve} achieved\indexnoun{achieved} achievement's\indexnoun{achievement's} achievement\indexnoun{achievement} achievements\indexnoun{achievements} achiever\indexnoun{achiever} achieves\indexnoun{achieves} achieving\indexnoun{achieving} Achilles\indexnoun{Achilles} aching\indexnoun{aching} achromatic\indexnoun{achromatic} achy\indexnoun{achy} acid\indexnoun{acid} acidification\indexnoun{acidification} acidify\indexnoun{acidify} acidity\indexnoun{acidity} acidophil\indexnoun{acidophil} acidophiles\indexnoun{acidophiles} acidoses\indexnoun{acidoses} acidosis\indexnoun{acidosis} acidulous\indexnoun{acidulous} Ackerman\indexnoun{Ackerman} acknowledge\indexnoun{acknowledge} acknowledgeable\indexnoun{acknowledgeable} acknowledged\indexnoun{acknowledged} ACM\indexnoun{ACM} acme\indexnoun{acme} acne\indexnoun{acne} acolyte\indexnoun{acolyte} aconite\indexnoun{aconite} acorn\indexnoun{acorn} acoustic\indexnoun{acoustic} acoustical\indexnoun{acoustical} acoustician\indexnoun{acoustician} acoustics\indexnoun{acoustics} acquaint\indexnoun{acquaint} acquaintance\indexnoun{acquaintance} acquaintanceship\indexnoun{acquaintanceship} acquainted\indexnoun{acquainted} acquiesce\indexnoun{acquiesce} acquiescence\indexnoun{acquiescence} acquiescent\indexnoun{acquiescent} acquirable\indexnoun{acquirable} acquire\indexnoun{acquire} acquirement\indexnoun{acquirement} acquisition\indexnoun{acquisition} acquisitions\indexnoun{acquisitions} acquisitive\indexnoun{acquisitive} acquisitiveness\indexnoun{acquisitiveness} acquit\indexnoun{acquit} acquittal\indexnoun{acquittal} acquittance\indexnoun{acquittance} acquitter\indexnoun{acquitter} acre\indexnoun{acre} acreage\indexnoun{acreage} acrid\indexnoun{acrid} acridity\indexnoun{acridity} acridness\indexnoun{acridness} acrimonious\indexnoun{acrimonious} acrimoniousness\indexnoun{acrimoniousness} acrimony\indexnoun{acrimony} acrobat\indexnoun{acrobat} acrobatic\indexnoun{acrobatic} acrobatics\indexnoun{acrobatics} acronym\indexnoun{acronym} acrophobia\indexnoun{acrophobia} acropolis\indexnoun{acropolis} across\indexnoun{across} acrostic\indexnoun{acrostic} acrylate\indexnoun{acrylate} acrylic\indexnoun{acrylic} act's\indexnoun{act's} act\indexnoun{act} actinic\indexnoun{actinic} actinide\indexnoun{actinide} actinium\indexnoun{actinium} actinometer\indexnoun{actinometer} action\indexnoun{action} actionable\indexnoun{actionable} actioned\indexnoun{actioned} actioning\indexnoun{actioning} activate\indexnoun{activate} activated\indexnoun{activated} activating\indexnoun{activating} activation\indexnoun{activation} activator\indexnoun{activator} active\indexnoun{active} actively\indexnoun{actively} activeness\indexnoun{activeness} activism\indexnoun{activism} activity\indexnoun{activity} Acton\indexnoun{Acton} actor\indexnoun{actor} actress\indexnoun{actress} actual\indexnoun{actual} actuality\indexnoun{actuality} actuarial\indexnoun{actuarial} actuary\indexnoun{actuary} actuate\indexnoun{actuate} actuation\indexnoun{actuation} actuator\indexnoun{actuator} acuity\indexnoun{acuity} acumen\indexnoun{acumen} acupressure\indexnoun{acupressure} acupuncture\indexnoun{acupuncture} acute\indexnoun{acute} acuteness\indexnoun{acuteness} acyclic\indexnoun{acyclic} acyclovir\indexnoun{acyclovir} ad's\indexnoun{ad's} ad\indexnoun{ad} adage\indexnoun{adage} adagio\indexnoun{adagio} Adair\indexnoun{Adair} Adam\indexnoun{Adam} adamant\indexnoun{adamant} Adamski\indexnoun{Adamski} Adamson\indexnoun{Adamson} adapt\indexnoun{adapt} adaptability\indexnoun{adaptability} adaptation\indexnoun{adaptation} adapted\indexnoun{adapted} adaptive\indexnoun{adaptive} adaptivity\indexnoun{adaptivity} adaptor\indexnoun{adaptor} ADC\indexnoun{ADC} add-on\indexnoun{add-on} add\indexnoun{add} addend\indexnoun{addend} addenda\indexnoun{addenda} addendum\indexnoun{addendum} addict\indexnoun{addict} addiction\indexnoun{addiction} Addis\indexnoun{Addis} Addison\indexnoun{Addison} addition\indexnoun{addition} additional\indexnoun{additional} additive\indexnoun{additive} additivity\indexnoun{additivity} addle\indexnoun{addle} addorsed\indexnoun{addorsed} address\indexnoun{address} addressed\indexnoun{addressed} addressee\indexnoun{addressee} addresses\indexnoun{addresses} addressing\indexnoun{addressing} adduce\indexnoun{adduce} adducer\indexnoun{adducer} adduct\indexnoun{adduct} adduction\indexnoun{adduction} adductor\indexnoun{adductor} Adelaide\indexnoun{Adelaide} Adele\indexnoun{Adele} Aden\indexnoun{Aden} Adenauer\indexnoun{Adenauer} adenine\indexnoun{adenine} adenoid\indexnoun{adenoid} adenoidal\indexnoun{adenoidal} adenoma\indexnoun{adenoma} adenomata\indexnoun{adenomata} adenoviral\indexnoun{adenoviral} adenovirus\indexnoun{adenovirus} adept\indexnoun{adept} adeptness\indexnoun{adeptness} adequacy\indexnoun{adequacy} adequate\indexnoun{adequate} adequateness\indexnoun{adequateness} adhere\indexnoun{adhere} adherence\indexnoun{adherence} adherent\indexnoun{adherent} adhesion\indexnoun{adhesion} adhesive\indexnoun{adhesive} adhesiveness\indexnoun{adhesiveness} adiabatic\indexnoun{adiabatic} adieu\indexnoun{adieu} adipose\indexnoun{adipose} Adirondack\indexnoun{Adirondack} adiós\indexnoun{adiós} adjacency\indexnoun{adjacency} adjacent\indexnoun{adjacent} adjectival\indexnoun{adjectival} adjective\indexnoun{adjective} adjoin\indexnoun{adjoin} adjourn\indexnoun{adjourn} adjudge\indexnoun{adjudge} adjudicate\indexnoun{adjudicate} adjudication\indexnoun{adjudication} adjudicator\indexnoun{adjudicator} adjunct\indexnoun{adjunct} adjuration\indexnoun{adjuration} adjure\indexnoun{adjure} adjust\indexnoun{adjust} adjusted\indexnoun{adjusted} adjustor's\indexnoun{adjustor's} adjusts\indexnoun{adjusts} adjutant\indexnoun{adjutant} Adkins\indexnoun{Adkins} Adler\indexnoun{Adler} adman\indexnoun{adman} admen\indexnoun{admen} admin\indexnoun{admin} adminicle\indexnoun{adminicle} adminicular\indexnoun{adminicular} administer\indexnoun{administer} administrable\indexnoun{administrable} administrate\indexnoun{administrate} administration\indexnoun{administration} administrator\indexnoun{administrator} administratrix\indexnoun{administratrix} admirable\indexnoun{admirable} admiral\indexnoun{admiral} admiralty\indexnoun{admiralty} admiration\indexnoun{admiration} admire\indexnoun{admire} admissibility\indexnoun{admissibility} admissible\indexnoun{admissible} admission\indexnoun{admission} admit\indexnoun{admit} admittance\indexnoun{admittance} admitted\indexnoun{admitted} admixture\indexnoun{admixture} admonish\indexnoun{admonish} admonisher\indexnoun{admonisher} admonition\indexnoun{admonition} admonitory\indexnoun{admonitory} ado\indexnoun{ado} adobe\indexnoun{adobe} adolescence's\indexnoun{adolescence's} adolescence\indexnoun{adolescence} adolescent\indexnoun{adolescent} Adolph\indexnoun{Adolph} Adonis\indexnoun{Adonis} adopt\indexnoun{adopt} adopted\indexnoun{adopted} adoption\indexnoun{adoption} adopts\indexnoun{adopts} adorable\indexnoun{adorable} adorableness\indexnoun{adorableness} adoration\indexnoun{adoration} adore\indexnoun{adore} adorn\indexnoun{adorn} adorned\indexnoun{adorned} adrenal\indexnoun{adrenal} adrenalin\indexnoun{adrenalin} adrenaline\indexnoun{adrenaline} Adrian\indexnoun{Adrian} Adriatic\indexnoun{Adriatic} Adrienne\indexnoun{Adrienne} adrift\indexnoun{adrift} adroit\indexnoun{adroit} adroitness\indexnoun{adroitness} ads\indexnoun{ads} ADSL\indexnoun{ADSL} adsorb\indexnoun{adsorb} adsorbate\indexnoun{adsorbate} adsorbent\indexnoun{adsorbent} adsorption\indexnoun{adsorption} ADte\indexnoun{ADte} adulate\indexnoun{adulate} adulation\indexnoun{adulation} adulator\indexnoun{adulator} adult\indexnoun{adult} adulterant\indexnoun{adulterant} adulterate\indexnoun{adulterate} adulterated\indexnoun{adulterated} adulteration\indexnoun{adulteration} adulterer\indexnoun{adulterer} adulteress\indexnoun{adulteress} adulterous\indexnoun{adulterous} adultery\indexnoun{adultery} adulthood\indexnoun{adulthood} adumbrate\indexnoun{adumbrate} adumbration\indexnoun{adumbration} advance\indexnoun{advance} advantage\indexnoun{advantage} advantageous\indexnoun{advantageous} advantageousness's\indexnoun{advantageousness's} advantageousness\indexnoun{advantageousness} advent\indexnoun{advent} Adventist's\indexnoun{Adventist's} adventist\indexnoun{adventist} adventitious\indexnoun{adventitious} adventure's\indexnoun{adventure's} adventure\indexnoun{adventure} adventures\indexnoun{adventures} adventuresome\indexnoun{adventuresome} adventuress\indexnoun{adventuress} adventurism\indexnoun{adventurism} adventurous\indexnoun{adventurous} adventurously\indexnoun{adventurously} adventurousness\indexnoun{adventurousness} adverb\indexnoun{adverb} adverbial\indexnoun{adverbial} adversarial\indexnoun{adversarial} adversary\indexnoun{adversary} adverse\indexnoun{adverse} adverseness\indexnoun{adverseness} adversity\indexnoun{adversity} advert\indexnoun{advert} advertise\indexnoun{advertise} advertised\indexnoun{advertised} advertising\indexnoun{advertising} advice\indexnoun{advice} advisabilities\indexnoun{advisabilities} advisability's\indexnoun{advisability's} advisability\indexnoun{advisability} advisable\indexnoun{advisable} advise\indexnoun{advise} advisedly\indexnoun{advisedly} advisee\indexnoun{advisee} advisor\indexnoun{advisor} advisory\indexnoun{advisory} advocacy\indexnoun{advocacy} advocate\indexnoun{advocate} adware\indexnoun{adware} adze\indexnoun{adze} Aegean\indexnoun{Aegean} aegis\indexnoun{aegis} aegrotat\indexnoun{aegrotat} Aeneas\indexnoun{Aeneas} Aeneid\indexnoun{Aeneid} aeolian\indexnoun{aeolian} Aeolus\indexnoun{Aeolus} aeon\indexnoun{aeon} aerate\indexnoun{aerate} aeration\indexnoun{aeration} aerator\indexnoun{aerator} aerial\indexnoun{aerial} aerialist\indexnoun{aerialist} aerie\indexnoun{aerie} aero\indexnoun{aero} aero-engine\indexnoun{aero-engine} aeroacoustic\indexnoun{aeroacoustic} aerobatic\indexnoun{aerobatic} aerobic\indexnoun{aerobic} aerodrome\indexnoun{aerodrome} aerodynamic\indexnoun{aerodynamic} aerodynamics\indexnoun{aerodynamics} aeroelastic\indexnoun{aeroelastic} aeroelasticity\indexnoun{aeroelasticity} aerofoil\indexnoun{aerofoil} aerogel\indexnoun{aerogel} aerogramme\indexnoun{aerogramme} aerolite\indexnoun{aerolite} aeromagnetic\indexnoun{aeromagnetic} aeromedical\indexnoun{aeromedical} aeromodeller\indexnoun{aeromodeller} aeromodelling\indexnoun{aeromodelling} aeronautic\indexnoun{aeronautic} aeronautical\indexnoun{aeronautical} aeronautics\indexnoun{aeronautics} aerophone\indexnoun{aerophone} aeroplane\indexnoun{aeroplane} aeroponic\indexnoun{aeroponic} aeroponically\indexnoun{aeroponically} aerosol\indexnoun{aerosol} aerospace\indexnoun{aerospace} Aeschylus\indexnoun{Aeschylus} Aesculapius\indexnoun{Aesculapius} Aesop\indexnoun{Aesop} aesthete\indexnoun{aesthete} aesthetic\indexnoun{aesthetic} aestheticism\indexnoun{aestheticism} aestival\indexnoun{aestival} aestivate\indexnoun{aestivate} aether\indexnoun{aether} aetiology\indexnoun{aetiology} AFAIK\indexnoun{AFAIK} afar\indexnoun{afar} AFC\indexnoun{AFC} affability\indexnoun{affability} affable\indexnoun{affable} affair\indexnoun{affair} affect\indexnoun{affect} affectation\indexnoun{affectation} affected\indexnoun{affected} affectedly\indexnoun{affectedly} affecter\indexnoun{affecter} affecting\indexnoun{affecting} affection\indexnoun{affection} affectionate\indexnoun{affectionate} affectionately\indexnoun{affectionately} affective\indexnoun{affective} affects\indexnoun{affects} afferent\indexnoun{afferent} affiance\indexnoun{affiance} affidavit\indexnoun{affidavit} affiliate\indexnoun{affiliate} affiliated\indexnoun{affiliated} affiliation\indexnoun{affiliation} affine\indexnoun{affine} affinity\indexnoun{affinity} affirm\indexnoun{affirm} affirmation\indexnoun{affirmation} affirmed\indexnoun{affirmed} affirms\indexnoun{affirms} affix\indexnoun{affix} afflatus\indexnoun{afflatus} afflict\indexnoun{afflict} affliction\indexnoun{affliction} affluence\indexnoun{affluence} affluent\indexnoun{affluent} afford\indexnoun{afford} affordable\indexnoun{affordable} afforest\indexnoun{afforest} afforestation\indexnoun{afforestation} affray\indexnoun{affray} affricate\indexnoun{affricate} affrication\indexnoun{affrication} affricative\indexnoun{affricative} affright\indexnoun{affright} affront\indexnoun{affront} Afghan\indexnoun{Afghan} Afghani\indexnoun{Afghani} Afghanistan\indexnoun{Afghanistan} aficionado\indexnoun{aficionado} afield\indexnoun{afield} afire\indexnoun{afire} aflame\indexnoun{aflame}


\chaptersubtitle{视频索引测试文字}
\printindex[video]

\chaptersubtitle{副标题副标题副标题副标题副标题}
\chapter*{章节名称章节名称章节名称章节名称}
\lipsum
%\cite{41}\cite{42}\cite{43}\cite{44}\cite{45}\cite{46}\cite{47}\cite{48}\cite{49}\cite{50}
\lipsum
%\cite{51}\cite{52}\cite{53}\cite{54}




\newcounter{wxcounter}
\newcommand{\wx}{\underline{~~\small\thewxcounter\refstepcounter{wxcounter}~~}}

\begin{Reading}{文章标题}{5}{3}
\lipsum[2]
\end{Reading}



\begin{Cloze}{文章标题}{5}{4.5}
\setcounter{wxcounter}{1}
	I made up my mind to drive to South Carolina to meet my friends in my used car. Though I had only been there once \wx and did not know the, \wx very well. I was on the \wx after I had made some inquiries.\par
	At Ashvelle, there was a crossroad where I could go on along the main road or I could take a shortcut. The short cut was to \wx several hills and was dangerous, I hesitated for a little while and then chose the main road, for I wanted to be \wx.\par
	Something strange happened after I drove a long, \wx and found it was not the correct road that Iwanted to \wx, but the hilly road I decided to avoid .I realized that it was at the. \wx that I had made the. \wx mistake. ``What shall I do ?" I asked myself. If I went back to take that road again, it would be very lateby the time I got to Columbia. Thin it \wx, I decided to go on. ``If \wx people can go along this road, why can't I ?" I \wx myself.\par
	The short cut, to my surprise, was not that \wx. In fact, it was only a very peaceful country road, \wx upand down two low \wx.There was \wx traffic. On both sides of the road, you could see trees, wild flowers, and \wx{} with cows and horses, My fear was \wx with the wind. Listening to the beautiful country musicover my car stereo, I drove on and \wx the scenery which was so quiet and so natural. Even my used car forgot to give me. \wx. It was just in this light heartedness that I arrived at my destination. My friends, after they heard what had happened to me, all said it sounded like an adventure.
\begin{QsNum}
\item \xx{before}{ago}{already}{still}
\item \xx{town}{country}{friends}{way}
\item \xx{train}{car}{highway}{phone}
\item \xx{have}{go}{ride}{cross}
\item \xx{safe}{dangerous}{fast}{slow}
\item \xx{moment}{way}{road}{day}
\item \xx{come}{leave}{take}{drive}
\item \xx{crossroad}{corner}{station}{beginning}
\item \xx{direction}{road}{disappointed}{interesting}
\item \xx{about}{over}{of}{up}
\item \xx{another}{the other}{other}{others}
\item \xx{asked}{forced}{encouraged}{told}
\item \xx{far}{safe}{dangerous}{dirty}
\item \xx{going}{coming}{driving}{walking}
\item \xx{lands}{cars}{farms}{hills}
\item \xx{heavy}{little}{few}{light}
\item \xx{farms}{trucks}{houses}{villages}
\item \xx{together}{gone}{covered}{coming}
\item \xx{looked}{liked}{enjoyed}{found}
\item \xx{happiness}{scenery}{joys}{problems}
\end{QsNum}
\end{Cloze}

\begin{Article}{乱数假文}{5}{1}
\zhlipsum\zhlipsum
\end{Article}

\begin{Paracol}
\begin{Vocabulary}{词汇}
\lipsum[2][1-5]
\tcbsubtitle{派生词}
\lipsum[3][1-5]
\tcbsubtitle{例句}
\lipsum[3][1-5]
\end{Vocabulary}
\end{Paracol}



\section*{不编号节不编号节不编号节不编号节不编号节不编号节不编号节不编号节不编号节}









%\printbib{Reference}

\printnomenclature

\end{Appendix}








\backmatter

\chaptersubtitle{测试文字测试文字}
\printindex[video]



\chaptersubtitle{测试文字测试文字}
\printindex[noun]


\chaptersubtitle{测试文字}
\chapter*{后记}
\lipsum
%\cite{1}\cite{2}\cite{3}\cite{4}\cite{5}\cite{6}\cite{7}\cite{8}\cite{9}\cite{10}
\lipsum
%\cite{11}\cite{12}\cite{13}\cite{14}\cite{15}\cite{16}\cite{17}\cite{18}\cite{19}\cite{20}


\chapter{后记}
\lipsum
%\cite{21}\cite{22}\cite{23}\cite{24}\cite{25}\cite{26}\cite{27}\cite{28}\cite{29}\cite{30}
\lipsum
\cite{31}\cite{32}\cite{33}\cite{34}\cite{35}\cite{36}\cite{37}\cite{38}\cite{39}\cite{40}









\begin{box6}{自选盒子}
\lipsum[2][1-3]
\end{box6}


\begin{TCBCODE}
\begin{box6}{自选盒子}
\lipsum[2][1-3]
\end{box6}
\end{TCBCODE}


\chaptersubtitle{参考文献}
\printbib*{Reference}






\makeback[][barcode.pdf]




\makespine[1.02cm]

\end{document}



















