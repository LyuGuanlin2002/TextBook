\documentclass[color=blue]{textbook-cn}%,splitbib,openany
\makeatletter
%%设计封面%%
\newcommand{\ifetdef}{
\maxsizebox{\paperwidth}{!}{\ifdefvoid{\@englishtitle}{}{\@englishtitle}}
}
\newcommand{\makecover}{
\tikzset{coverup node/.style={text=ColorA,anchor=north west,inner sep=0mm},
coverdown node/.style={text=white,anchor=west,inner sep=0mm},
opacity node/.style={anchor=north west,font=\FontSize{100pt}\bfseries\sffamily,text=ColorB!25}
}
\begin{titlepage}
\begin{tikzpicture}[remember picture,overlay]
\coordinate(A)at(current page.north west);
\coordinate(B)at(current page.north east);
\coordinate(C)at(current page.south east);
\coordinate(D)at(current page.south west);
\coordinate(E)at([yshift=5.25cm]current page.west);
\coordinate(F)at([yshift=5.25cm]current page.east);
\coordinate(G)at($0.5*(E)+0.5*(C)$);
%%背景1%%
\draw[ColorB!5,fill=ColorB!5,sharp corners](D)rectangle(B);
%%半透明文字%%
\node[opacity node,text opacity=1.0](O_1)at(A){\ifetdef};
\foreach \x[evaluate=\x as \botpoint using int(\x-1),
evaluate=\x as \opa using 1.2-0.2*\x] in {2,...,10}{
\node[opacity node,text opacity=\opa](O_\x)at(O_\botpoint.south west){\ifetdef};}
%%背景2%%
\draw[ColorA!80,fill=ColorA!80](E)rectangle(C);
%%封面图%%
\node[inner sep=0mm,anchor=center](CI)at(G){\ifdefvoid{\@coverimage}{}{\includegraphics[width=\paperwidth,height=\paperheight/2+5.25cm]{\@coverimage}}};
%%标题%%
\node[font=\FontSize{55pt}\bfseries\sffamily,coverup node](T)at(current page text area.north west){\@title};
%%英文标题%%
\node[font=\FontSize{20pt}\bfseries\sffamily,coverup node](ET)at([yshift=-0.5cm]T.south west){\ifdefvoid{\@englishtitle}{}{\@englishtitle}};
%%副标题%%
\node[font=\FontSize{30pt}\bfseries\sffamily,coverup node](ST)at([yshift=-0.5cm]ET.south west){\ifdefvoid{\@subtitle}{}{\@subtitle}};
%%英文副标题%%
\node[font=\FontSize{15pt}\bfseries\sffamily,coverup node](EST)at([yshift=-0.5cm]ST.south west){\ifdefvoid{\@englishsubtitle}{}{\@englishsubtitle}};
%%作者%%
\node[font=\FontSize{15pt}\bfseries\sffamily,coverdown node](A)at([yshift=4.0cm]current page text area.west){主编\quad\@author};
%%出版社%%
\node[coverdown node](PL)at(current page text area.south west){\ifdefvoid{\@presslogo}{}{\includegraphics[width=0.75cm]{\@presslogo}}};
\node[font=\FontSize{15pt}\bfseries\sffamily,coverdown node](PN)at([xshift=0.25cm]PL.east){\ifdefvoid{\@pressname}{}{\@pressname}};
\end{tikzpicture}
\end{titlepage}}

%%设计封底%%
\newcommand{\ifestdef}{
\maxsizebox{\paperwidth}{!}{\ifdefvoid{\@englishsubtitle}{}{\@englishsubtitle}}
}
\NewDocumentCommand{\makeback}{oo}{
\tikzset{backup node/.style={text=ColorA,anchor=north east,inner sep=0mm,text width=\textwidth},
backdown node/.style={text=white,anchor=west,inner sep=0mm},
opacity node/.style={anchor=north east,font=\FontSize{100pt}\bfseries,text=ColorB!25}}
\begin{titlepage}
\begin{tikzpicture}[remember picture,overlay]
\coordinate(A)at(current page.north west);
\coordinate(B)at(current page.north east);
\coordinate(C)at(current page.south east);
\coordinate(D)at(current page.south west);
\coordinate(E)at([yshift=5.25cm]current page.west);
\coordinate(F)at([yshift=5.25cm]current page.east);
\coordinate(G)at($0.5*(E)+0.5*(C)$);
%%背景1%%
\draw[ColorB!5,fill=ColorB!5,sharp corners](D)rectangle(B);
%%半透明文字%%
\node[opacity node,text opacity=1.0](O_1)at(B){\ifestdef};
\foreach \x[evaluate=\x as \botpoint using int(\x-1),
evaluate=\x as \opa using 1.2-0.2*\x] in {2,...,10}{
\node[opacity node,text opacity=\opa](O_\x)at(O_\botpoint.south east){\ifestdef};}
%%背景2%%
\draw[ColorA!80,fill=ColorA!80](E)rectangle(C);
%%封底图%%
\node[inner sep=0mm,anchor=center](BI)at(G){\ifdefvoid{\@backimage}{}{\scalebox{1}[1]{\includegraphics[width=\paperwidth,height=\paperheight/2+5.25cm]{\@backimage}}}};
%%条形码%%
\IfValueTF{#2}{
\node[draw=white,fill=white,inner xsep=0mm,anchor=south east](BC)at([yshift=-0.5cm]current page text area.south east){\includegraphics[width=5.0cm]{#2}};}{\relax}
%%说明文字%%
\node[font=\FontSize{12pt}\bfseries\sffamily,backup node](BT)at(current page text area.north east){\hspace*{2.0em}#1};
%%书籍系列%%
\node[font=\FontSize{15pt}\bfseries\sffamily,backdown node](S)at([yshift=4.0cm]current page text area.west){\ifdefvoid{\@series}{}{\@series}};
%%书籍系列%%
\node[font=\FontSize{15pt}\bfseries\sffamily,backdown node](V)at([yshift=-0.5cm]S.south west){\ifdefvoid{\@version}{}{第{\@version}版}};
\end{tikzpicture}
\end{titlepage}}

%%设计书脊%%
\NewDocumentCommand{\makespine}{O{1.0cm}}{
\newlength{\SpineWidth}
\setlength{\SpineWidth}{#1}
\tikzset{spine title/.style={font=\FontSize{20pt}\bfseries\sffamily,text=white,text width=1.0cm,align=center},
spine subtitle/.style={font=\FontSize{15pt}\bfseries\sffamily,text=white,text width=1.0cm,align=center},
spine englishtitle/.style={font=\FontSize{20pt}\bfseries\sffamily,text=ColorA,inner sep=0mm}}
\begin{titlepage}
\begin{tikzpicture}[remember picture,overlay]
\coordinate(A)at(current page.north);
\coordinate(B)at(current page.south);
\coordinate(C)at([yshift=5.25cm]current page.center);
%%背景%%
\draw[ColorB!5,fill=ColorB!5]
([xshift=-\SpineWidth/2]A)rectangle([xshift=\SpineWidth/2]B);
\draw[ColorA!80,fill=ColorA!80]
([xshift=-\SpineWidth/2]C)rectangle([xshift=\SpineWidth/2]B);
%%书脊英文标题%%
\node[spine englishtitle,anchor=north](ET)at([yshift=-0.5cm]A){\ifdefvoid{\@englishtitle}{}{\rotatebox[origin=c]{-90}{\maxsizebox{\paperheight/2-5.25cm}{\SpineWidth}{\@englishtitle}}}};
%%书脊标题%%
\node[spine title,anchor=north](T)at([yshift=-0.5cm]C){\@title};
%%书脊副标题%%
\node[spine subtitle,anchor=north](ST)at([yshift=-0.5cm]T.south){\ifdefvoid{\@subtitle}{}{\@subtitle}};
%%书脊出版社名称%%
\node[spine subtitle,anchor=south](PN)at([yshift=0.5cm]B){\ifdefvoid{\@pressname}{}{\@pressname}};
%%书脊出版社logo%%
\node[spine subtitle,anchor=south](PL)at([yshift=0.25cm]PN.north){\ifdefvoid{\@presslogo}{}{\includegraphics[width=0.8cm]{\@presslogo}}};
\end{tikzpicture}
\end{titlepage}}
\makeatother
\makeatletter


\NewDocumentCommand{\Logo}{O{0.5}mmm}{
\def\globalscale{#1}
\begin{tikzpicture}[y=1.0pt,x=1.0pt,yscale=-\globalscale,xscale=\globalscale,every node/.append style={scale=\globalscale},inner sep=0pt,outer sep=0pt]
\path[draw,draw opacity=0,left color=ColorB,right color=ColorA,even odd rule,line width=0pt,miter limit=8.0] (291.5,56.6) .. controls (372.9,58.3) and (460.0,93.2) .. (530.6,159.7) .. controls (572.1,198.8) and (603.1,244.2) .. (622.9,291.6) -- (623.6,293.4) -- (616.4,280.9).. controls (600.2,254.0) and (579.9,228.7) .. (555.6,205.8) .. controls (426.2,83.9) and (232.6,79.3) .. (123.1,195.5) .. controls (78.7,242.7) and (55.0,302.6) .. (51.1,365.2) -- (50.7,377.5) -- (47.8,365.6) .. controls (28.4,279.0) and (46.0,192.0) .. (103.8,130.6) .. controls (152.0,79.5) and (219.5,55.1) .. (291.5,56.6) -- cycle;
\draw[white,line width={\globalscale*10.0pt},fill=ColorH](600,270)circle(50);
\node[font=\bfseries\sffamily\FontSize{90},text=white,inner sep=0mm](P)at(600,270){\maxsizebox{1.5cm}{!}{#4}};
\node[font=\bfseries\sffamily\FontSize{90},text=ColorH,anchor=north,inner xsep=0mm,minimum height=2.0cm](T)at(315,190){\maxsizebox{12.0cm}{!}{\MakeUppercase{#2}}};
\node[font=\bfseries\sffamily\FontSize{150},text=ColorA,anchor=north,inner xsep=0mm,minimum height=4.0cm](S)at([yshift=30]T.south){\maxsizebox{16.0cm}{!}{\MakeUppercase{#3}}};
\draw[ColorB,line width={\globalscale*10.0pt},line cap=round]([xshift=-20,yshift=20]S.south west)--([xshift=20,yshift=20]S.south east);
\end{tikzpicture}}

\makeatother


\usepackage{zhlipsum,lipsum}%乱数假文(测试文字)
\graphicspath{{./figure/}{./figures/}{./image/}{./images/}{./graphics/}{./graphic/}{./pictures/}{./picture/}}%提供多种图片路径

%%设置符号表样式%%
\nomensetup{columnsep=2.0em,columns=3}
\makenomenclature

%%设置索引样式%%
\indexsetup{}
\makeindex[name=noun,title=名词索引,columns=3,columnsep=2.0em,options=-s indexstyle.ist]
\makeindex[name=video,title=视频索引,columnsep=2.0em,options=-s indexstyle.ist]

%%设置参考文献样式%%
\bibsetup{intoc}

%%定义罗马数字%%
\makeatletter
\newcommand{\rmnum}[1]{\romannumeral #1}
\newcommand{\Rmnum}[1]{\expandafter\@slowromancap\romannumeral #1@}
\makeatother

%%设置书籍信息%%
\series{物理学系列教程}
\version{三}
\title{物理学基础讲义}
\englishtitle{Fundamental of Physics}
\subtitle{经典力学}
\englishsubtitle{Classical Mechanics}
\author{RQCEN}
\pressname{BILIBILI教育出版社}
\presslogo{gj.pdf}
\coverimage{coverimage.pdf}%
\backimage{coverimage.pdf}%

\newcommand{\indexnoun}[1]{\index[noun]{#1}}
\newcommand{\indexvideo}[1]{\index[video]{#1}}

\newcommand{\zhindexnoun}[2]{\zhindex[noun]{#1}{#2}}
\newcommand{\zhindexvideo}[2]{\zhindex[video]{#1}{#2}}


\newcommand{\inputplus}{\raisebox{-0.5em}{\Logo[0.07]{physics}{PLUS}{\faPlus}}~~~}
\newcommand{\inputlook}{\raisebox{-0.5em}{\Logo[0.07]{physics}{LOOK}{\faPlay}}~~~}
\newcommand{\inputlisten}{\raisebox{-0.5em}{\Logo[0.07]{physics}{LISTEN}{\faVolumeDown}}~~~}
\newcommand{\inputread}{\raisebox{-0.5em}{\Logo[0.07]{physics}{READ}{\faStickyNote}}~~~}


%%选择题的4个选项,根据选项内容长度自动排版%%
\colorlet{ListColor}{ColorA}
\newlength{\la}\newlength{\lb}\newlength{\lc}\newlength{\ld}
\newlength{\lhalf}\newlength{\lquarter}\newlength{\lmax}
\newcommand{\choice}[4]{
\settowidth{\la}{A.~#1~~~}\settowidth{\lb}{B.~#2~~~}
\settowidth{\lc}{C.~#3~~~}\settowidth{\ld}{D.~#4~~~}
\ifthenelse{\lengthtest{\la>\lb}}
{\setlength{\lmax}{\la}}{\setlength{\lmax}{\lb}}
\ifthenelse{\lengthtest{\lmax<\lc}}
{\setlength{\lmax}{\lc}}{\relax}
\ifthenelse{\lengthtest{\lmax<\ld}}
{\setlength{\lmax}{\ld}}{\relax}
\setlength{\lhalf}{0.5\linewidth}
\setlength{\lquarter}{0.25\linewidth}
\ifthenelse{\lengthtest{\lmax>\lhalf}}
{\begin{hlist}[pre skip=0pt,item skip=0pt,item offset={1.5em},label={\textbf{\color{ListColor}{\Alpha{hlisti}.}}},pre label={}]1
\hitem #1
\hitem #2
\hitem #3
\hitem #4
\end{hlist}}
{\ifthenelse{\lengthtest{\lmax>\lquarter}}
{\begin{hlist}[pre skip=0pt,item skip=0pt,item offset={1.5em},label={\textbf{\color{ListColor}{\Alpha{hlisti}.}}},pre label={}]2
\hitem #1
\hitem #2
\hitem #3
\hitem #4
\end{hlist}}
{\begin{hlist}[\parskip=0pt,pre skip=0pt,item skip=0pt,item offset={1.5em}, label={\textbf{\color{ListColor}{\Alpha{hlisti}.}}},pre label={}]4
\hitem #1
\hitem #2
\hitem #3
\hitem #4
\end{hlist}}
}}




%%盒子代码与实际效果对比%%
\newtcblisting{TCBCODE}{enhanced standard,breakable,
before=\par\noindent\medskip,after=\par\medskip,
colback=white,colbacklower=white,colframe=ColorH,
before lower=\bigskip,rounded corners,arc=5.0pt,boxrule=2.0pt}

\newcommand{\E}{\mathbb{E}}
\renewcommand{\Pr}{\mathbb{P}}
\newcommand{\EP}{\mathbb{E}^{\mathbb{P}}}
\newcommand{\EQ}{\mathbb{E}^{\mathbb{Q}}}
\newcommand{\dif}{\,{\rm d}}
\newcommand{\Var}{{\rm Var}}
\newcommand{\Cov}{{\rm Cov}}


%%%%%%%%%%%%%%%%%%%%%%%%%%%%%%%%%%%%%%%%%%%%%%%%%%%%%%%%%%%%%%%%%%%%%%%%%%%%%%%%%%%
\begin{document}
\MainFont





\makeatletter
%\@mainmattertrue
\makeatother

%\part{独立测试}

\chapter{独立测试}

\section{独立测试}
\lipsum



\makecover


\maketitle

\partimage{Sunset.jpg}
\part{测试测试测试}

\lipsum


\frontmatter

\lipsum


\input{preface.tex}


\chaptersubtitle{Brief Contents}
\shorttableofcontents{0}


\chaptersubtitle{Contents}
\tableofcontents{1}

\mainmatter


\lipsum
\begin{Definition}[定义名称]
\lipsum[1]
\end{Definition}


\addcontentsline{toc}{part}{不编号部分目录测试}


\Example*{\lipsum[2]}

\Example{\lipsum[2]}

\Variant{\lipsum[2]}

\Example{\lipsum[2]}

\Variant{\lipsum[2]}

\makeatother



\ExerciseTitle{练习题练习题练习题练习题练习题练习题练习题练习题练习题练习题}


\ExerciseTitle{练习题}




\ExerciseTitle{ExercisePPpppqqq}



\lipsum[1]
\Example{\lipsum[1]}

\Answer{\lipsum[2]}


\begin{Definition}[定义名称]
\lipsum[1]
\end{Definition}

\begin{Lemma}[定义名称]
	\lipsum[2]
\end{Lemma}

\begin{Lemma}[定义名称]
	\lipsum[2]
\end{Lemma}

\ref{Lem.3.11.1}
\pageref{Lem.3.11.2}

\begin{Proposition}[定义名称]
	\lipsum[2]
\end{Proposition}


\begin{Corollary}[定义名称]
	\lipsum[2]
\end{Corollary}


\begin{Definition}[定义名称]
	\lipsum[2]
\end{Definition}


\chapter{测试章测试章测试章测试章测试章测试章测试章测试章测试章测试章测试章测试章测试章测试章测试章测试章测试章测试章测试章测试章测试章}
\lipsum

\begin{Project}

\makeatletter

%\chapter{节环境中测试章1}

	
\section{测试课题题目1}

\begin{Definition}[定义名称]
\lipsum[1]
\end{Definition}

\begin{Theorem}[定理名称]
\lipsum[1]
\end{Theorem}
%
\begin{Axiom}[公理名称]
\lipsum[1]
\end{Axiom}

\begin{Lemma}[引理名称]
\lipsum[1]
\end{Lemma}

\begin{Corollary}[推论名称]
\lipsum[1]
\end{Corollary}

\begin{Proposition}[命题名称]
\lipsum[1]
\end{Proposition}

\begin{Thinking}
\lipsum[2]
\end{Thinking}





\begin{Thinking*}
	\lipsum[2]
\end{Thinking*}


\section{测试课题题目2}

\begin{Definition}[定义名称]
\lipsum[1]
\end{Definition}

%\chapter{节环境中测试章2}

\Example{\lipsum[2]}

%\chapter*{节环境中测试章2}

\section{测试课题题目3}

\begin{Definition}[定义名称]
\lipsum[1]
\end{Definition}




\makeatother

\end{Project}







\begin{Project}

\makeatletter

\section{测试课题题目q}


%\chapter{节环境中测试章3}


\section{测试课题题目4}


\begin{Case}
	\item\lipsum[1][1]
	\item\lipsum[1][1]
	\item\lipsum[1][1]
	\item\lipsum[1][1]
\end{Case}


\begin{Definition}[定义名称]
	
	\lipsum[1]
\end{Definition}











\section{测试课题题目5}

\begin{Definition}[定义名称]
\lipsum[1]
\end{Definition}


%\chapter{节环境中测试章4}


\section{测试课题题目6}


\lipsum[2]

\begin{Definition}[定义名称]
\lipsum[1]
\end{Definition}

\subsection*{对比测试文字}

\begin{box1}{易错点警示测试文字测试文字测试文字易错点警示测试文字测试文字测试文字易错点警示测试文字}[测试标题测试标题测试标题]
\lipsum[2]
\end{box1}


\begin{box1}{易错点警示}
\lipsum[2]
\end{box1}

\begin{Warning}
	\lipsum[2]
\end{Warning}


\begin{box1}{WARNING}[注意注意注意]
	\lipsum[2]
\end{box1}


\begin{box1}{}[标题]
	\lipsum[2][1-3]
\end{box1}

\begin{box1}{ }[标题]
	\lipsum[2][1-3]
\end{box1}

\begin{box1}{标签}[]
	\lipsum[2][1-3]
\end{box1}

\begin{box1}{标签}[ ]
	\lipsum[2][1-3]
\end{box1}

\begin{box1}{ }[ ]
	\lipsum[2][1-3]
\end{box1}

\begin{box1}{}[]
	\lipsum[2][1-3]
\end{box1}

\begin{box1}{标签}[标题]
	\lipsum[2][1-3]
\end{box1}




\begin{box2}{演示与实验}[标题标题]
	\lipsum[2]
\end{box2}


\begin{box2}{演示与实验演示与实验演示与实验演示与实验演示与实验演示与实验演示与实验演示与实验演示与实验演示与实验演示与实验}[标题标题]
	\lipsum[2]
\end{box2}

\begin{box2}{演示与实验演示与实验演示与实验演示与实验演示与实验演示与实验}[标题标题]
	\lipsum[2]
\end{box2}


\begin{box2}{ }[ ]
	\lipsum[2]
\end{box2}


\newpage
\begin{box3}{演示与实验演示与实验演示与实验演示与实验演示与实验演示与实验演示与实验演示与实验演示与实验演示与实验演示与实验}
	\lipsum[2]
\end{box3}


\begin{box3}{ }[]
	\lipsum[2][1-3]
\end{box3}





\begin{box4}{引理}
\lipsum[2]
\end{box4}

\begin{box4}{引理引理引理引理引理引理引理引理引理引理引理引理引理引理引理引理}
\lipsum[2]
\end{box4}



\begin{box5}{拓展链接}
\lipsum[2]
\end{box5}

\begin{box5}{拓展链接拓展链接拓展链接拓展链接拓展链接拓展链接拓展链接拓展链接拓展链接}
	\lipsum[2]
	\tcbsubtitle{测试标题}
	\lipsum[2]
\end{box5}



\begin{box6}{资料卡片}
	\lipsum[2]
\end{box6}



\begin{box6}{ }[]
	\lipsum[2][1-3]
\end{box6}


\begin{box6}{资料卡片资料卡片资料卡片资料卡片资料卡片资料卡片资料卡片资料卡片资料卡片资料卡片资料卡片}
	\lipsum[2]
\end{box6}


\begin{Information}
	\lipsum[2]
	\tcbsubtitle{测试标题}
	\lipsum[2]
\end{Information}




\RelaInfo{\zhlipsum[3]{\bfseries 测试文字}}


\begin{Paracol}
\RelaInfo{\zhlipsum[3]}
\switchcolumn
\RelaInfo{\bfseries\lipsum[3][1-3]}

\end{Paracol}



\begin{box8}{阅读理解}[Reading Comprehension][ColorB]
\lipsum
\end{box8}


\begin{box8}{阅读理解阅读理解阅读理解阅读理解阅读理解阅读理解阅读理解阅读理解阅读理解}[][ColorB]
	\lipsum
\end{box8}




\begin{Vocabulary}
	\lipsum[2]
	\tcbsubtitle{派生词}
	\lipsum[2]
\end{Vocabulary}




\begin{PythonBox}{代码}
import
\end{PythonBox}



\begin{box0}[left=0mm]{本章要点}[][ColorA]
\lipsum[3]
\end{box0}

\begin{box0}[left=0mm]{ }[][ColorA]
	\lipsum[3]
\end{box0}

\begin{box0}[left=0mm]{测试文字测试文字测试文字测试文字测试文字测试文字测试文字测试文字测试文字}[][ColorA]
	\lipsum[3]
\end{box0}



\begin{Exercise}
\lipsum[2]
\end{Exercise}


\Example{\lipsum[2]}



\begin{Proof}
	\lipsum[2]
\end{Proof}





\makeatother

\end{Project}





\chapter{环境外测试章节}

\begin{Exercise}
	\lipsum[2]
\end{Exercise}

\begin{Thinking}
	\lipsum[2]
\end{Thinking}


\begin{Improve}
	\lipsum[2]
\end{Improve}



\begin{Definition}[定义名称]
	\lipsum[2]
\end{Definition}



{\ttfamily\bfseries\itshape 测试文字}
{\kaiti\itshape 测试文字}
{\kaiti gan}

\begin{Topic}

\section{小专题1}


\Example{\zhlipsum[1]}

\subsection{}
\subsection{  }



\begin{Exercise}
	\lipsum[2]
\end{Exercise}

\begin{Thinking}
	\lipsum[2]
\end{Thinking}


\begin{Improve}
	\lipsum[2]
\end{Improve}




\makeatletter
%\noindent\@egextitle{  }{ColorA}{  }
%\@egextitle{}{ColorA}{}\\
%\@egextitle{Label}{ColorA}{  }
%\@egextitle{Label}{ColorA}{}\\
%\@egextitle{  }{ColorA}{Name}
%\@egextitle{}{ColorA}{Name}\\
%\@egextitle{Label}{ColorA}{Name}
%\@egextitle{标签}{ColorA}{题目}

\makeatother


%\chapter{专题测试章节专题测试章节专题测试章节专题测试章节专题测试章节专题测试章节专题测试章节专题测试章节专题测试章节专题测试章节专题测试章节专题测试章节专题测试章节专题测试章节专题测试章节专题测试章节}

\zhlipsum


\section{小专题2}


\Example{\zhlipsum[1]}


\Example{\zhlipsum[2]}

	
\end{Topic}


\begin{Test}
\chapter{测试题}


\begin{Definition}[定义名称]
	\lipsum[2]
\end{Definition}

\section[主观题]{主观题主观题主观题主观题主观题主观题主观题主观题主观题主观题主观题主观题主观题主观题主观题主观题}


%\section*{主观题主观题主观题主观题主观题主观题主观题主观题主观题主观题主观题主观题主观题主观题主观题主观题}


\subsection{单项选择题}

\lipsum

\end{Test}



\begin{Appendix}
\chapter[WWWW]{中文}

\section[子附录测试]{小标题小标题}
\section[子附录测试]{小标题小标题qp}

\section[子附录测试]{abaqus}


\section[子附录测试]{ABAQUS}


\zhlipsum
	
\chapter{英文}
\zhlipsum
\end{Appendix}

\partimage{Mountain.jpg}
\part{测试章节测试章节测试章节测试章节测试章节测试章节测试章节}
\part[测试章节]{测试章节测试章节测试章节测试章节测试章节}

\lipsum\lipsum


\part{测试测试测试}

%\lipsum


\backmatter

\lipsum

%\stopcontents[part]
\partimage{Sunset.jpg}
\part*{AAAAAAA-测试文字}
\partimage{Mount.jpg}
\partintro{\zhlipsum[2]}
\part{测试章节}
\part[BabyBabyBaby]{AAAAABBBBB-CCCCCC-DDDD}


\lipsum
\makeatletter























\lipsum[2]


\makeatother









\begin{Test}

\chapter[PPPPPLLL]{中文}
\thepart\zhlipsum

\section{backmatter测试节标题}
\section{测试节标题}

\subsection{backmatter测试小节标题}


\begin{Definition}[定义名称]
	\lipsum[2]
\end{Definition}




\chapter{英文}
\zhlipsum
\end{Test}



\chaptersubtitle{PostScript}
\chapter{后记}
\thechapter
\zhlipsum\zhlipsum


\begin{Definition}[定义名称]
	\lipsum[1]
\end{Definition}




\chapter[我爱你]{后记}
\thechapter
\zhlipsum\zhlipsum
\chapter*{后记}



\chapter*[我讨厌你]{Lipsum}
\zhlipsum\zhlipsum

\chapter*{AAAAAAAAAAAAAAA测试一下}

\zhlipsum\zhlipsum

\chapter*{AAA}

\begin{Thinking}
	\lipsum[2]
\end{Thinking}
\zhlipsum\zhlipsum

\part[宝贝]{测试章节}


\lipsum

\begin{Definition}
\lipsum[2]
\end{Definition}

\begin{Lemma}[定义名称]
	\lipsum[2]
\end{Lemma}



\begin{Topic}

\section{小专题3}
%\chapter{专题测试章节}
\section{LipsumLipsumLipsum}

\lipsum

\end{Topic}


\makeatletter
%\@SectionStarStyle{第abA节第abA节第abA节}{QQQAAABjjjjjl}
%\@SectionStarStyle{第abA节第abA节}{QQQAAABjjjjjl}
%\@SectionStarStyle{第abA节第abA节第abA节}{节测试文字测试文字测试文字测试文字测试文字测试文字测试文字QQQAAABjjjjjl}
%\@SectionStarStyle{第abA节第abA节第abA节}{节测试文字测试文字测试文字测试文字测试文字测试文字测试文字}
%\@SectionStarStyle{第第节第节}{测试文字测试文字测试文字测试文字测试文字测试文字测试文字QQQAAABjjjjjl}
%\@SectionStarStyle{第第节第节第第节第节第第节第节第第节第节第第节第节}{测试文字测试文字测试文字测试文字测试文字测试文字测试文字QQQAAABjjjjjl}
%\@SectionStarStyle{第第节第节第第节第节第第节第节第第节第节第第节第节}{测试文字测试文字测试文字测试文字}
%\@SectionStarStyle{测试文字}{测试文字}
\@SectionStarStyle{第第节第}{测试文字测试文字测试文字测试文字测试文字测试文字测试文字测试文字测试文字}
%\@SectionStarStyle{}{测试文字测试文字测试文字}
%\@SectionStarStyle{第第节第}{}
%\@SectionStarStyle{第第节第第第节第节第第节第节第第节第节第第节第节第第节第节第第节第节第第节第节第第节第节第第节第节第第节第节第第节第节第第节第节第第节第节第第节第节第第节第节第第节第节第第节第节第第节第节第第节第节第第节第节第第节第节第第节第节第第节第节第第节第节第第节第节第第节第节第第节第节第第节第节第第节第节第第节第节第第节第节第第节第节第第节第节第第节第节第第节第节}{}
\@SectionStyle{A}{Q}{测试文字测试文字测试文字测试文字测试文字测试文字测试文字测试文字}{1}
\@SectionStyle{Lipsum}{ajhf}{测试文字Q测试文字}{1}
\@SectionStyle{AGRT}{QQWAE}{测试文字Q测试文字}{1}
\@SectionStyle{第}{节}{测试文字测试文字}{1}

\@SectionStyle{}{节}{测试文字测试文字}{1}
\@SectionStyle{第}{}{测试文字测试文字}{1}
\@SectionStyle{第测试文字测试文字}{测试文字测试文字}{测试文字测试文字}{1}


\@SectionStyle{练习}{}{测试文字测试文字}{1}
\@SectionStyle{练习}{}{}{1}

\makeatother









\cite{1}\cite{2}


\chaptersubtitle{Reference}
\printbib{./index/reference}



\makeback[][barcode.pdf]




%\makespine[1.02cm]

\end{document}



















